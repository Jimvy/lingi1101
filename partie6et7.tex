%Packages à ajouter pour la compilation
%\usepackage{amsmath}
%\usepackage{lmodern}
%\usepackage{vmargin}
%\usepackage{tabularx}
%\usepackage[usenames,dvipsnames]{color}
%Partie 6
\chapter{La logique des prédicats}
 
\section{Introduction}

Nous allons maintenant étudier une logique beaucoup plus expressive que la
logique propositionnelle, la logique des prédicats, qui est aussi appelée
la logique de premier ordre.\footnote{Il existe des logiques d'ordres supérieures,
mais elles ne feront pas l'objet de ce cours.}
Voici un premier tableau qui montre les différences entre la logique propositionnelle vue jusqu'à présent et la logique des prédicats que nous allons étudier.
\begin{center}
\begin{tabular}{|c|c|}
\hline 
Logique Propositionnelle & Logique des prédicats \\ 
\hline
Propositions premières & Prédicats P(x,y) \\ 
P, Q, R & Quantifieurs: $\exists$x, $\forall$y \\ 
$\hookrightarrow$ Pas de Relations & $\hookrightarrow$ Relation \\ 
\hline 
\end{tabular} 
\end{center}

On note P(x,y) dans la logique des prédicats avec x,y, les arguments du prédicat P qui sont des variables.
Dans la logique propositionnelle, chaque proposition est isolée/indépendante alors que dans les prédicats on peut lier plusieurs prédicats ensemble.


\begin{center}
\begin{tabular}{|c|c|c|}
\hline 
Exemple & Logique propositionnelle & Logique des prédicats \\ 
\hline 
Socrate est un philosophe & P & Phil(Socrate) \\ 
Platon est un philosophe & Q & Phil(Platon) \\ 
\hline 
\end{tabular} 
\end{center}

En logique propositionnelle il n'y a aucunes relations entre P et Q, alors qu'en logique des prédicats on peut lier Socrate et Platon avec le prédicat Philosophe qui prend en argument le nom du philosophe (Socrate ou Platon dans ce cas). Phil(Socrate) est donc vrai. On peut donc dire grâce aux prédicats que Socrate et Platon sont "la même chose", des philosophes.\\

Un autre exemple de prédicat:

\begin{center}
$\forall \alpha$ Phil($\alpha$) $\Rightarrow$ Savant($\alpha$)\\
\vspace{3mm}
$\hookrightarrow$ ... \textit{cette formulation permet de résumer un très grand nombre de faits. L'ensemble des arguments $\alpha$ peut être infini}
\end{center}
Comme Socrate est un philosophe, on peut déduire que Socrate est un savant aussi!

Dire la même chose en logique propositionnelle serait beaucoup plus compliqué: \\

"Socrate est un savant" Proposition "R"\\
\indent "Platon est un savant" Proposition "S"\\

On va donc noter en logique propositionnelle
\begin{center}
(P$\Rightarrow$R) $\cup$ (Q$\Rightarrow$S) $\cup$ ...(\textit{potentiellement infini})
\end{center}

On doit tout énumérer car il n'y a aucune relation entre les différentes propositions. S'il y a un nombre infini, ça ne marche pas. Il y a donc de grandes limitations dans la logique propositionnelle.

Néanmoins parfois la logique propositionnelle peut être utile. 
Il existe des outils informatiques qui utilisent la logique propositionnelle. On peut prendre l'exemple de "SAT solver" à qui on donne des équations booléennes très compliquées et qui va trouver les valeurs des propositions primitives qui rendent vraie cette proposition. C'est donc assez utilisé! La logique propositionnelle est utile, mais si l'on veut faire du raisonnement sur plus que  "vrai" et "faux" avec des relations entre des propositions,  la logique propositionnelle ne marche pas. Si on veut faire un logiciel qui montre une certaine intelligence, il faut utiliser la logique des prédicats.\\

Autre exemple:

\begin{tabular}{|ccc|} 
\hline
Exemple & Logique propositionnelle & Logique des prédicats \\ 
\hline
Tout adulte peut voter & P & $\forall$x adulte(x) $\Rightarrow$ voter(x) \\ 
John est un adulte & Q & adulte(\textcolor{OliveGreen}{John}) \\ 
\line(1,0){50} & \line(1,0){10} & \line(1,0){45} \\ 

John peut voter & \textcolor{Red}{?R?}& voter(\textcolor{OliveGreen}{John}) \\ 
\hline
\end{tabular}\\

Ce genre de raisonnement est très difficile à faire en logique propositionnelle alors qu'en logique des prédicats c'est beaucoup plus simple! Le John en ligne 3 et en ligne 4 correspond à la même personne, ou de manière plus général à la même variable! 
Ceci montre donc bien qu'il nous faut la logique des prédicats pour faire des relations de ce type.

\section{Quantificateurs}

Les expressions "pour tout $x$" ($\forall x$) et "il existe $x$ tel que" ($\exists x$) sont appelés des quantificateurs en logique des prédicats. Les quantificateurs permettent d'instancier les variables dans une formule. La notion de portée d'un quantificateur est un concept très important auquel il faut faire très attention, car il peut changer complètement le sens d'une formulation. \\

$\forall x$ (enfants($x$) $\wedge$ intelligents($x$) $\Rightarrow$ $\exists y$ aime($x$,$y$)) \\

$\forall x$ (enfants($x$) $\wedge$ intelligents($x$)) $\Rightarrow$ $\exists y$ aime($x$,$y$) \\

Ces deux formules peuvent paraître équivalentes, mais en réalité elles ont un sens tout à fait différent.
En effet, dans le deuxième cas on remarque que le quantificateur $\forall x$ ne porte pas sur la dernière variable $x$ qui est en argument du prédicat aime($x$,$y$).

Il faut donc faire bien attention à quel quantificateur une variable s'identifie lorsqu'on manipule des formules. \\

\begin{itemize}
\item[$\bullet$] $\forall x$ P($x$) $\wedge$ $\exists x$ Q($x$) : contient deux variables différentes\\
 
\item[$\bullet$] $\forall x$ $\exists x$  P($x$) $\wedge$ Q($x$) : est une forme incorrecte, conflit des noms de variables \\
\end{itemize}

Pour résoudre ces conflits, on fait appel à une nouvelle opération, le renommage. Cette opération permet de changer le nom des variables tout en conservant le sens de la formule. Ainsi on obtient : \\

 $\forall x$ $\exists z$  P($x$) $\wedge$ Q($z$)  \textit{renommage (2)} \\
 
 Le concept de variables, de leurs portées ainsi que d'opérateurs en logique des prédicats fait fortement penser au langage de programmation
 
Une comparaison entre un code et une formule est tout à fait envisageable. Prenons un code tout à fait banal comprenant des variables différentes avec des portées différentes qui ont le même identificateur ainsi qu'une formule correspondante. 

\begin{verbatim}
1.  begin {
2.      var x,y: int;        
3.      x := 4;
4.      y := 2;
5.    
6.      begin {
7.          var x: int;
8.          x := 5;
9.          x := x*y;
10.     end }
11.     x := x*y;
12. end }
\end{verbatim}


\textcolor{Green}{$\forall x$} \textcolor{Red}{$\forall y$} p(\textcolor{Green}{$x$}) $\wedge$ (\textcolor{Blue}{$\exists x$}  q(\textcolor{Blue}{$x$},\textcolor{Red}{$y$}) $\vee$ r(\textcolor{Green}{$x$},\textcolor{Red}{$y$}))  \\ 

En analysant morceau par morceau de la formule : \\

\begin{itemize}

\item[$\bullet$] " \textcolor{Green}{$\forall x$} \textcolor{Red}{$\forall y$} p(\textcolor{Green}{$x$}) $\wedge$ " correspond aux points $\lbrace 2,3,4 \rbrace$ du code \\

\item[$\bullet$]" \textcolor{Blue}{$\exists x$}  q(\textcolor{Blue}{$x$},\textcolor{Red}{$y$}) $\vee$ " correspond aux points $\lbrace 7,8,9 \rbrace$ \\
 
\item[$\bullet$]" r(\textcolor{Green}{$x$},\textcolor{Red}{$y$})) "  correspond au point $\lbrace 11 \rbrace$ \\

\end{itemize}

Cet exemple illustre parfaitement la ressemblance et le lien entre le monde de la programmation et celui de la logique des prédicats

\section{Syntaxe}

\begin{tabular}{|c|c|c|}
	\hline
	Symboles logiques & quantificateurs & $\forall$ $\exists$ \\
	                  & connecteurs logiques & $\wedge$ $\vee$ $\neg$ $\Rightarrow$ $\Leftrightarrow$ \\
	                  & parenthèses & ( ) \\
	                  & variables & $x, y, z$ \\
	                  & true, false & \\
	\hline
	Symboles non logiques & symboles de prédicats & $P$ $\varphi$ $R$ + arguments $\geq 0$ \\
						  & symboles de fonction & + $arguments \geq 0 $\\
	\hline
\end{tabular}

\section{Grammaire}
\subsection{Règles de formation}
\begin{tabular}{rl}
$<formule>::=$ 	  &	$<formule$ $atomique>$ \\
				  & $\vert$ $\neg$ $<formule>$ \\
				  & $\vert$ $<formule>$ $<connecteur>$ $<formule>$ \\
				  & $\vert$ $\forall <var>.<formule>$ \\
				  & $\vert$ $\exists <var>.<formule>$ \\
$<formule$ $atomique>::=$ 
				  & true, false \\
				  & $\vert$ $<predicat>(<terme>*)$ \\
$<terme>::=$	  & $<constante>$ \\
				  & $\vert <var>$ \\
				  & $\vert <fonction>(<terme>*)$ \\
$<connecteur$ $binaire>::=$ 
				  & $\wedge \vert \vee \vert \Rightarrow \vert \Leftrightarrow$ \\

\end{tabular}

\section{Exemple de sémantique}
Dans la logique des prédicats, nous gardons les notions de modèle et d'interprétation déjà définis dans la logique propositionnelle. Même si la logique des prédicats est beaucoup plus puissante, sa sémantique reste similaire à la logique propositionnelle.
Comme pour la logique propositionnelle, une interprétation peut avoir une valeur (true, false).

Illustrons par un exemple : 

\begin{center}
$p : P(b,f(b)) \Rightarrow \exists y   P(a,y)$  \\
\vspace{3mm}
\end{center}
On suppose que le $a$ et le $b$ sont des constantes et que le $f$ est une fonction. Il faut donner un sens à cette formule et pour cela il faut donner un sens à : 
\begin{itemize}
\item[$\bullet$]P : $val_{I}(P) = $ $ \geq $ \hspace{3mm} (\textit{considéré comme un vrai prédicat})
\item[$\bullet$] a : $val_{I}(a) = $ $ 2 $ 
\item[$\bullet$] b : $val_{I}(b) = $ $ \pi $ 
\item[$\bullet$] $f$ : $val_{I}(f) = $ $ f_{i} $ \hspace{3mm} $f_{i}= \Re \rightarrow \Re : d \rightarrow \dfrac{d}{2} $ 
\end{itemize}
Avec ces éléments, on peut donc trouver l'interprétation : 
\begin{center}
\textit{Si $\pi \geq  \dfrac{\pi}{2}$, alors $\exists$ $ d \in \Re$ tel que $\sqrt2 \geq d$ }
\end{center}
Cette phrase n'est pas très utile, mais avec cette interprétation, la phrase logique donne ce sens. Une autre interprétation donnerait un sens totalement différent à la phrase logique. Voici une autre interprétation totalement différente :
\begin{itemize}
\item[$\bullet$] a : $val_{I}(a) = $ "Barack Obama"
\item[$\bullet$] b : $val_{I}(b) = $ "Vladimir Putin"
\item[$\bullet$] $f$ : $val_{I}(f) = $ $ f_{i} $ \hspace{3mm} $f_{i} \rightarrow$ père(d) 
\item[$\bullet$] P :  $val_{I}(P) = $ $P_{I}$ \hspace{3mm} $d_{1}$ est enfant de $d_{2}$\\
\end{itemize}

Avec ce nouveau sens, on trouve l'interprétation suivante : 
\begin{center}
\textit{Si Vladimir Putin est l'enfant du père de Vladimir Putin alors $\exists$ une personne telle que Barack Obama est l'enfant de cette personne.}
\end{center}
La seconde interprétation est très différente de la première malgré le fait que ce soit la même formule à l'origine ! La connexion entre une formule et son sens permet de garder une certaine souplesse dans le sens où l'on peut choisir ça. C'est un peu comme dans la logique propositionnelle, mais avec encore plus de souplesse.

On peut se demander si ces deux interprétations sont des modèles de la formule ? \\
La première interprétation est un modèle de la formule, car le sens de la formule est vrai dans l'interprétation. En effet, $\pi \geq \dfrac{\pi}{2}$ et $\exists$ $ d \in \Re$ tel que $\sqrt2 \geq d$. On voit donc que le modèle est vrai.

La deuxième interprétation est aussi un modèle de la formule, car l'interprétation trouvée est vraie aussi. Cela peut paraître bizarre, mais c'est correct. 
%Partie 7
\section{Détails techniques}
La première question que l'on se pose est: comment faire des preuves en logique des prédicats ? Il faut noter que les quantificateurs $\forall$ et $\exists$ rendent le raisonnement plus subtil. Il faudra des règles pour pouvoir raisonner sur ces quantificateurs.
\subsection{Sémantique}
La sémantique en logique des prédicats est très proche de celle en logique propositionnelle. Cependant, l'interprétation, I, est plus précise qu'en logique propositionnelle et est également plus compliquée à utiliser à cause des variables et des symboles de fonction. Cette interprétation peut être décrite comme une paire: $I = pair(D_{I}, val_{I}) $ avec D le domaine de discours I et la fonction val qui est l'interprétation de tous les symboles. $D_{I} \ne \emptyset $ $ \forall s \in S$ avec s soit un symbole de prédicat soit une fonction. $val_{I}(S) = P_{I}$ une fonction $P_{I}:D_{I}^{n} \rightarrow (True,False)$ avec S symbole d'une fonction. Il existe aussi une vraie fonction $f_{I}:D_{I}^{n} \rightarrow D_{I}$ avec n le nombre d'arguments tel que $val_{I}(S) = f_{I} $. Cela implique que dans le domaine de discours chaque fonction correspond à une vraie fonction et chaque prédicat correspond à un vrai prédicat. 


Si l'on rajoute une variable, x par exemple, l'expression devient: $ (var, x) \rightarrow val_{I}(x) = x_{I} \in D_{I}$. L'interprétation d'une variable, $x_{I}$ est un élément (n'importe lequel en fonction de l'interprétation) de $D_{I}$. La fonction $VAL_{I}$ est la même fonction, mais pour les formules et pas uniquement pour les symboles comme avant. Néanmoins, on ne doit pas redéfinir $,VAL_{I}$ car elle existe à partir du moment ou $val_{I}$ et $D_{I}$ existent. Définition: $VAL_{I}: TERM \cup  PRED \rightarrow D_{I} \cup (True, False) $ avec TERM l'ensemble des termes et PRED l'ensemble des
prédicats (toutes les formules en logique des prédicats). Les termes peuvent être définis comme $ t \rightarrow VAL_{I}(t)$ et les prédicats comme $ P \rightarrow VAL_{I}(P)$ avec P une formule. Il y a une forte relation entre $val_{I} $ et $VAL_{I}: val_{I}((P(T_{1},...,T_{n})) = P_{I}(VAL_{I}(t_{1}),...,VAL_{I}(t_{n}))$ avec $ P_{I} = VAL_{I}(P)$ 
La première partie de l'égalité est un prédicat avec des arguments tandis que la seconde est formée de plusieurs prédicats avec un argument. 


Définissons d'autres formules: $VAL_{I}(P \wedge Q) $ est true si $VAL_{I}(P) = True$ et $VAL_{I}(Q) = True$. false sinon. Donc il faut $VAL_{I}(\forall x.P) = True$ avec P une formule pouvant dépendre de x (il faut remplacer toutes les occurrences de x dans P et la formule doit rester vraie. Si pour chaque d $\in D_{I}$ $ I' = I \cup (x \leftarrow d)$ et $val_{I}(x) = d$ alors l'interprétation de x doit être: $VAL_{I}'(P) = True$.
Pour $VAL_{I}(\exists x.P) = True$ le raisonnement est identique en remplaçant $\forall$ par $\exists$.
\section{Différence avec la logique des propositions}
Les trois différences entre les deux logiques sont:
\begin{enumerate}
\item les variables,
\item les quantificateurs,
\item les symboles de fonction (moins important).
\end{enumerate}  
Imaginons un modèle $B:
\{
  \begin{array}{rcr}
    P_{1},...P_{n}
  \end{array}
\}
$
Si nous utilisons une interprétation I pour B cela donne:
$\forall P_{I} \in B : VAL_{I} (P_{I}) = True$ qui est très générale, car $P_{I}$ peut avoir des variables, des quantificateurs ...

% partie 1 (15 à 30 min) Thomas)

\section{Preuves avec règles}

    Il est possible de faire des preuves avec des règles de preuve. Une preuve est une simple manipulation de symboles qui sont des règles. Mais il faut justifier ces règles pour obtenir des résultats vrais. Elles sont justifiées en raisonnant sur les interprétations pour vérifier si elles sont correctes ou non. Tout cela est la sémantique. C'est la base pour pouvoir faire des preuves.

Beaucoup de choses restent les mêmes que dans la logique des propositions. La preuve est toujours un objet mathématique, une séquence avec des formules, des justifications, une application des règles. Elle commence avec des prémisses et finit par une conclusion. Il est possible de faire des preuves manuelles, mais aussi des preuves automatisées. C'est une généralisation de l'approche de la logique des propositions. 

Il y a toujours :
\begin{itemize}
  \item Une règle de résolution pour les preuves automatisées, mais celle-ci est plus générale. Elle va utiliser un concept appelé "unification". Ce nouveau concept est introduit à cause des variables. En effet, celles-ci peuvent être différentes, il faut donc trouver un nouveau moyen de les fusionner.
  \item Une forme normale, qui  est plus compliquée à cause des quantificateurs et des symboles de fonctions, mais qu'il est encore possible de l'obtenir.
  \item un algorithme avec ses propriétés. Mais il est moins fort/complet que l'algorithme développé pour la logique des propositions. Il ne sera plus décidable, mais seulement semi-décidable. C'est-à-dire que parfois il tournera en boucle. Cela est dû aux variables et aux quantificateurs. La logique des prédicats est beaucoup plus riche que la logique des propositions, mais en contrepartie l'algorithme arrive à prouver moins de choses. Cependant, l'algorithme sera toujours adéquat, mais pas forcément complet. Il ne sera pas toujours possible de trouver une preuve, même quand elle existe parce qu'elle sera trop compliquée.
\end{itemize}
Qu'est il possible de faire avec ce genre d'algorithme moins fort?

Il y a deux possibilités : 
\begin{itemize}
\item Un assistant de preuve\\
C'est un outil qui aide les gens à faire des preuves formelles. Deux exemples d'assistants de preuves sont Coq et Isabelle. C'est un outil très sophistiqué, mais qui a permis de prouver des choses de manière totalement formelle, alors qu'avant des preuves prenaient des dizaines voire des centaines de pages de preuves mathématiques. Mais cet assistant ne fait pas tout parce que l'algorithme est moins bon. Cependant, il aide beaucoup. C'est à l'être humain de lui donner des coups de pouce sous forme de lemmes, hypothèses, chemins, stratégies ... Ensuite, l'algorithme s'occupe de la manipulation des symboles.\\
Un exemple très célèbre est le théorème de la coloration d'une carte. La question est : est-il toujours possible de colorier chaque pays avec une couleur, de façon à ce que deux pays limitrophes n'aient pas la même couleur et en utilisant un certain nombre de couleurs différentes? Ce n'est pas évident à prouver et ça a demandé beaucoup de travail aux mathématiciens. Mais récemment, Georges Gonthier (un informaticien) a réussi à formuler ce problème avec l'assistant de preuves. Ce fut un tour de force. Désormais, il existe une preuve complètement formalisée, sans erreur pour ce théorème.
\item L'utiliser dans les langages de programmation\\
L'algorithme peut être considéré comme le moteur d'un programme. C'est ce qu'on appelle maintenant la programmation logique. Elle consiste à utiliser la logique dans un programme. Le langage le plus célèbre qui a suivi cette approche est Prolog. Ce fut un énorme succès, car les gens ne croyaient pas que c'était possible de faire un programme en logique qui pouvait tourner. Cela a donné naissance à la programmation par contraintes (une contrainte est une relation logique). Cette discipline est très utile pour les optimisations, par exemple dans le cas du "voyageur de commerce".
\end{itemize}
\subsubsection{}
La logique des prédicats n'est donc pas quelque chose de seulement théorique, destiné uniquement aux mathématiciens. Les gens ont vraiment essayé avec succès d'utiliser la logique dans l'exécution des programmes.
\subsubsection{}

%partie 3 de 30 à 45 min (Guillaume)
\chapter{Preuves en logique des prédicats}

On va généraliser l'approche de la logique propositionnelle, car comme vu précédemment le langage des prédicats est beaucoup plus riche.  Il ajoute entre autres :

\begin{itemize}
    \item Des variables
    \item Des constantes
    \item Des fonctions (détaillé plus tard)
    \item Des prédicats
    \item Des quantificateurs
\end{itemize}

Les preuves en logique des prédicats ressemblent très fort aux preuves en logique propositionnelle. Il y a encore des prémisses, des formules avec leurs justificatifs et une conclusion. On peut aussi utiliser des preuves indirectes et des preuves conditionnelles. Cela reste un objet formel.

\begin{center}
\fbox{
$
\begin{array}{l l l}
  1. & \fbox{\ldots} &  Prémisses \\
  2. & \fbox{Formule, regle} &  Justification \\
  \ldots & \ldots &  \ldots \\
  n. & \fbox{Conclusion} &  Justification \\
\end{array}
$
}
\end{center}
\section{Exemple}

La méthode pour passer de $\forall x \cdot P(x) \wedge Q(x)$ (prémisse) à $\forall x \cdot P(x)\wedge(\forall x \cdot Q(x))$ (conclusion) est la suivante:
\begin{center}
$
\begin{array}{l l}
  1. & $Enlever les quantificateurs pour avoir des variables libres$ \\
  2. & $Raisonner sur l'intérieur$\\
  3. & $Remettre les quantificateurs$\\
\end{array}
$
\end{center}
Les étapes difficiles à réaliser correctement sont les étapes $1$ et $3$. Voici la preuve en "français" :

En retirant les quantificateurs des prémisses, cela donne :
"Comme $P(x) \wedge Q(x)$ est vrai pour tout $x$, alors $P(x)$ est vrai pour tout $x$". 
De là, on peut remettre les quantificateurs pour obtenir $\forall x \cdot P(x)$. De façon similaire, on obtient $\forall x \cdot Q(x)$. Et on conclut en remettant les quantificateurs: $\forall x \cdot P(x) \wedge \forall x \cdot Q(x)$, en utilisant la conjonction.

En preuve formelle, cela donne :

\begin{center}
\fbox{
$
\begin{array}{l l l}
  1. &  \forall x \cdot P(x) \wedge Q(x) &  $Prémisses$ \\
  2. & P(x) \wedge Q(x) &  $Élimination de $\forall \\
  3. & P(x) &  $Simplification$ \\
  4. & \forall x \cdot  P(x) &  $Introduction de $ \forall \\
  5. & Q(x) &  $Simplification $ \forall \\
  6. & \forall x \cdot Q(x) &  $Introduction de $\forall \\
  7. & \forall x \cdot P(x) \wedge \forall x \cdot Q(x) &  $Conjonction$ \\
\end{array}
$
}
\end{center}

On a donc utilisé 4 règles en plus par rapport aux preuves formelles en logique propositionnelle (les règles de la logique propositionnelle restent valables en logique des prédicats):
\begin{itemize}
\item Élimination de $\forall$
\item Introduction de $\forall$
\item Élimination de $\exists$
\item Introduction de $\exists$
\end{itemize}

Certaines de ces règles sont simples d'utilisation, d'autres sont plus difficiles. Il est également possible d'utiliser d'autres règles (certaines plus générales que d'autres\footnote{Voir "Inference logic" ou "Predicate logic"}).

\begin{framed}
\textbf{Note} :\\

Il est possible d'utiliser les quantificateurs  dans les formules mathématiques. Typiquement, on ne les note pas, car ils sont présents de manière implicite. Par exemple :
\begin{itemize}
\item $\forall x \cdot \sin(2x) = 2 \cdot \sin(x) \cdot \cos(x)$ 
\item $\forall x \cdot x + x = 2x$
\item $\exists x \cdot \sin(x) + \cos(x) = 0,5$
\item $\exists x \cdot x + 5 = 9$
\end{itemize} 

On peut remarquer que pour les deux premiers cas, $x$ est une véritable variable, on peut donc ajouter un quantificateur universel $\forall$. 
Pour les deux cas suivants, on remarque que $x$ est une inconnue, car il y a une équation à résoudre et une solution à trouver, on peut donc ajouter un quantificateur existentiel $\exists$.\\

Dans certains cas, les quantificateurs existentiels et universels sont utilisés au sein de la même formule mathématique.
\begin{itemize}
\item[] $\forall a \cdot \forall b \cdot \forall c \cdot \exists x \cdot ax^{2}+bx+c = 0$
\end{itemize}
Dans l'exemple ci-dessus, nous avons 4 variables : $a,b,c,x$. Les 3 premières sont des véritables variables, on peut les affecter à n'importe quelle valeur, tandis que la dernière est une inconnue, c'est la solution à trouver. Il faut donc trouver $x$ pour toutes les valeurs possibles de $a,b,c$.
\end{framed}

\section{Règles en logique des prédicats}
En logique des prédicats, pour trouver une preuve, on va faire des manipulations de formules.

\subsection{La substitution}
Une manipulation fréquente en logique des prédicats est la \textbf{substitution}. Elle consiste à prendre une formule et remplacer une partie par une autre.\\

Si on a une formule $p[x/t]$ (où $p$ veut dire "toutes les formules"). On va remplacer toutes les occurrences libres de $x$ par $t$.\\
On peut aussi écrire cela de la manière suivante :\\
$p[x/t]$ possède deux portées :
\begin{itemize}
\item[] La première se note $p[x]$ et correspond à une partie de la règle.
\item[] La seconde se note $p[t]$ et correspond à l'autre partie de la règle.\\
\end{itemize}

Exemple :
\begin{center}
\begin{tabular}{|l |l |>{\raggedright}m{6cm}|}
\hline
1. &$P(x) \rightarrow \forall y \cdot (P(x) \wedge R(y))$&$[x/y]$ veut dire qu'on va remplacer toutes les occurrences libres de $x$ par $y$.\tabularnewline
\hline
2. &$P(y) \rightarrow \forall y \cdot (P(y) \wedge R(y))$&En remplaçant $x$ par $y$, on a changé le sens de la formule, car avant, $x$ n'était pas dans la portée du quantificateur alors que maintenant il l'est. Ce changement de sens s'appelle une \textbf{capture de variable}, car la variable $y$ est capturée par le quantificateur. Pour résoudre ce problème, on va effectuer un \textbf{renommage}.\tabularnewline
\hline
3. &$P(y) \rightarrow \forall z \cdot (P(y) \wedge R(z))$&Résultat après renommage (pour éviter la capture de variable).\tabularnewline
\hline
\end{tabular}
\end{center}
