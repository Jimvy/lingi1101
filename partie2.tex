
\chapter{La logique propositionnelle}

\begin{quote}
« \textit{\foreignlanguage{english}{They don't even need to know what they're
talking about.}} »\\--- Richard Feynman à propos des mathématiciens.
\end{quote}

La logique propositionnelle est la plus simple des formes de logique. Elle
permet de formaliser des connexions logiques entre des propositions.
Par exemple, prenons les expressions suivantes:

\begin{enumerate}
\item « S’il fait beau, alors je vais dehors. »
\item « Cet homme est grand et fort. »
\item « Il fait jour mais pas nuit. »
\end{enumerate}

Dans ces expressions, nous pouvons définir des propositions premières:

\begin{enumerate}
\item il fait beau ;
\item je vais dehors ;
\item cet homme est grand ;
\item cet homme est beau ;
\item il fait jour ;
\item il fait nuit.
\end{enumerate}

Le sens de ces propositions en langue naturelle n'a aucune importance
dans la logique propositionnelle. C'est
pourquoi elles seront remplacées par des lettres majuscules :

\begin{enumerate}
\item A = « il fait beau » ;
\item B = « je vais dehors » ;
\item C = « cet homme est grand » ;
\item D = « cet homme est beau » ;
\item E = « il fait jour » ;
\item F = « il fait nuit ».
\end{enumerate}

Une proposition logique est alors :

\begin{itemize}
\item soit une des propositions premières ;
\item soit une combinaison de propositions logiques connectées par des
  connecteurs logiques.
\end{itemize}

Ainsi, les exemples de propositions précédentes peuvent être réécrits
comme ceci (la signification précise des différents symboles sera décrite
plus loin) :

\begin{enumerate}
\item $A \Rightarrow B$ ;
\item $C \land D$ ;
\item $E \land \lnot F$.
\end{enumerate}

% TODO: Plus d’exemples des différents connecteurs pour tout illustrer ? % On ne trouve pas nécessaire dans la mesure ou les trois exemple sont représentatifs - Cyril et Jon

L’avantage de cette notation par rapport aux phrases en français est qu’elle
nous permet d’effectuer des raisonnements formels sur les propositions
logiques. En particulier, nous pouvons précisément définir :

\begin{enumerate}
\item une \textit{syntaxe} (définie par une grammaire)
qui définit ce qui est une proposition logique et ce qui ne l’est pas ;
\item une \textit{sémantique} qui donne un sens à chaque proposition logique ;
\item une \textit{théorie de preuve} permettant, en sachant qu’une proposition est
  vraie, de trouver d’autres propositions vraies (par exemple à partir de $A
  \Rightarrow B$ on peut trouver $\lnot B \Rightarrow \lnot A$).
\end{enumerate}

\section{La syntaxe}

La logique propositionnelle est un \textit{langage formel}. Ce langage peut être défini
à l’aide d’une grammaire sur un \textit{alphabet}. L’alphabet est l’ensemble des
symboles qui composent une proposition logique, c’est-à-dire :

\begin{itemize}
\item les lettres majuscules représentant les différentes propositions
  premières : $A$, $B$, $C$, etc. ;
\item $\true$ et $\false$ qui représentent des propositions qui sont
  respectivement toujours vraies et toujours fausses ;
\item les différents connecteurs logiques :
	\begin{itemize}
	\item Conjonction (« et ») : $\land$
	\item Disjonction (« ou ») : $\lor$
	\item Négation :  $\lnot$
	\item Implication :  $\Rightarrow$
	\item Équivalence :  $\Leftrightarrow$
	\end{itemize}
\item les caractères de ponctuation « $($ » et « $)$ ».
\end{itemize}

Cependant, toutes les séquences composées de ces caractères ne sont pas des
phrases propositionnelles. La grammaire suivante permet de donner les règles que
les phrases propositionnelles doivent respecter~:

\begin{tabular}{rrl}
  $\textrm{<identificateur>}$ & ::= & $A$ | $B$ | $C$ | $D$ | \dots \\
  $\textrm{<proposition>}$
  & ::= & $\true$ \\
  & | & $\false$ \\
  & | & $\textrm{<identificateur>}$ \\
  & | & $(\textrm{<proposition>})$ \\
  & | & $\lnot \textrm{<proposition>}$ \\
  & | & $\textrm{<proposition>} \land \textrm{<proposition>}$ \\
  & | & $\textrm{<proposition>} \lor \textrm{<proposition>}$ \\
  & | & $\textrm{<proposition>} \Rightarrow \textrm{<proposition>}$ \\
  & | & $\textrm{<proposition>} \Leftrightarrow \textrm{<proposition>}$
\end{tabular}
\vspace{2 mm}

Remarquez que seules les séquences de symboles qui respectent cette
grammaire sont des phrases propositionnelles. Ainsi, $p \Leftrightarrow q$
n’est \textit{pas} une phrase propositionnelle parce que les propositions
premières doivent \textit{toujours} être
représentées par des lettres majuscules ; de même les phrases en français —
ou en klingon, ou encore dans d’autres formalismes mathématiques – comme «
  s’il fait beau alors je vais dehors » ne respectent pas la grammaire
précédente et ne sont donc pas des phrases propositionnelles.

\paragraph{Métalangage}
Notez que notre discours (en français et en notation mathématique)
à propos des phrases propositionnelles n’est
pas une phrase propositionnelle. La grammaire précédente, la description de
l’alphabet, et cette explication en français parlent de propositions logiques sans en
être, et font donc partie de ce qui est appelé le \textit{métalangage}.
Un métalangage est un deuxième langage utilisé pour parler d'un premier langage.
Dans notre discours, le premier langage est la logique propositionnelle,
et le deuxième langage est le français augmenté avec des notations mathématiques.

Le concept de métalangage est important pour distinguer le raisonnement formel
(en utilisant les opérations logiques définies formellement) et
le raisonnement informel (typiquement en langage naturel augmenté par des notations mathématiques).
Le raisonnement informel reste très important, même si le but ultime est de faire le plus
possible en raisonnement formel, parce qu'il est plus facile d'éviter des erreurs
de raisonnement dans un raisonnement formel.

\section{Les tables de vérité}

La grammaire définie dans la section précédente permet d'écrire
les propositions en logique propositionnelle,
mais elle ne leur donne pas un sens,
c’est-à-dire de définir quand une proposition est vraie ou fausse.
Plus précisément, le sens d'une proposition s'appelle la sémantique
de la proposition.
Pour savoir si une proposition est vraie ou fausse, il faut commencer
par choisir pour chacune de ses propositions premières si elle est vraie ou fausse.
Ensuite on peut déterminer si la proposition est vraie ou fausse.
Il y a deux approches principales pour faire cela:
les {\em tables de vérité} et les {\em interprétations}.
Dans cette section nous expliquerons les tables de vérité.
Dans la section suivante nous expliquerons les interprétations.

Rappelez-vous que la signification des
propositions premières n’a aucune importance, le choix de sa véracité
est donc complètement
arbitraire.
Le choix qui décrit au mieux le monde réel n’est qu’un des choix
possibles parmi tous les autres.
On sait également que les propositions
$\true$ et $\false$ sont, respectivement, toujours vraies et toujours
fausses.

Les autres propositions sont construites à partir de propositions plus
simples. Leur véracité est fonction de celle des propositions qui les
composent. Prenons par exemple $p \land q$, où $p$ et $q$ sont d’autres
propositions. La véracité de $p \land q$ est une fonction de celle de $p$ et
de $q$ : $p \land q$ est vrai si et seulement si $p$ et $q$ sont vrais aussi
($\land$ est un « et » logique). Cette relation peut être exprimée à l’aide
de la table de vérité suivante :

\vspace{2 mm}
\begin{tabular}{ll|l}
  $p$ & $q$ & $p \land q$ \\
  \hline

  $\true$ & $\true$ & $\true$ \\
  $\true$ & $\false$ & $\false$ \\
  $\false$ & $\true$ & $\false$ \\
  $\false$ & $\false$ & $\false$
\end{tabular}

\vspace{2 mm}
Voici la table de vérité des autres connecteurs logiques :
\vspace{2 mm}

\begin{tabular}{ll|l}
  $p$ & $q$ & $p \lor q$ \\
  \hline

  $\true$ & $\true$ & $\true$ \\
  $\true$ & $\false$ & $\true$ \\
  $\false$ & $\true$ & $\true$ \\
  $\false$ & $\false$ & $\false$
\end{tabular}
\vspace{2 mm}

\begin{tabular}{ll|l}
  $p$ & $q$ & $p \Leftrightarrow q$ \\
  \hline

  $\true$ & $\true$ & $\true$ \\
  $\true$ & $\false$ & $\false$ \\
  $\false$ & $\true$ & $\false$ \\
  $\false$ & $\false$ & $\true$
\end{tabular}
\vspace{2 mm}

\begin{tabular}{l|l}
  $p$ & $\lnot p$ \\
  \hline

  $\true$ & $\false$ \\
  $\false$ & $\true$
\end{tabular}
\vspace{2 mm}

\begin{tabular}{ll|l}
  $p$ & $q$ & $p \Rightarrow q$ \\
  \hline

  $\true$ & $\true$ & $\true$ \\
  $\true$ & $\false$ & $\false$ \\
  $\false$ & $\true$ & $\true$ \\
  $\false$ & $\false$ & $\true$
\end{tabular}
\vspace{2 mm}

Remarquez que le dernier tableau est correct. Il n’est parfois pas intuitif
que la proposition $A \Rightarrow B$ (qui pourrait s’exprimer en français
par « si $A$, alors $B$ ») soit toujours vraie quand $A$ est faux, mais c’est
pourtant le cas : la proposition dit que quand $A$ est vrai, $B$ doit
l’être aussi, mais elle ne donne aucune information sur le cas où $A$ est
faux.

Les parenthèses, quant à elles, servent à distinguer des propositions telles
que $A \land (B \lor C)$ et $(A \land B) \lor C$, qui s’écriraient de la
même manière sans parenthèses alors qu’elles n’ont pas la même table de
vérité :

\begin{tabular}{lll|ll}
  $A$ & $B$ & $C$ & $A \land (B \lor C)$ & $(A \land B) \lor C$ \\
  \hline

  $\true$  & $\true$  & $\true$  & $\true$  & $\true$  \\
  $\true$  & $\true$  & $\false$ & $\true$  & $\true$  \\
  $\true$  & $\false$ & $\true$  & $\true$  & $\true$  \\
  $\true$  & $\false$ & $\false$ & $\false$ & $\false$ \\
  $\false$ & $\true$  & $\true$  & $\false$ & $\true$  \\
  $\false$ & $\true$  & $\false$ & $\false$ & $\false$ \\
  $\false$ & $\false$ & $\true$  & $\false$ & $\true$  \\
  $\false$ & $\false$ & $\false$ & $\false$ & $\false$
\end{tabular}

\section{Les interprétations}

Une autre façon de définir si une proposition est vraie ou fausse est d’utiliser
une \textit{interprétation}. Si $E_P$ est l’ensemble des propositions premières,
alors une interprétation $I$ définit la fonction $\val_I : E_P \rightarrow
\{\true, \false\}$ 
qui permet de savoir si ces propositions premières sont vraies ou fausses.\footnote{La notation
$f : A \rightarrow B$ signifie que $f$
est une fonction depuis l’ensemble $A$ vers l’ensemble $B$.  } 
Par exemple, on pourrait écrire ceci :

\[\val_I(A) = \true\]
\[\val_I(B) = \false\]
\[\val_I(C) = \true\]

Étant donné la fonction $\val_I$, il est possible de définir la fonction
$\VAL_I : P \rightarrow \{\true, \false\}$, qui est une extension de
$\val_I$ à $P$, l’ensemble de toutes les propositions. L’équivalent des
tables de vérité pourrait être des expressions telles que~:

\[\forall p \in E_P. \VAL_I(p) = \val_I(p)\]

% Les définitions du type « A \land B est vrai si A et B sont vrais » me
% paraissent être cycliques, donc je les ai plutôt écrites comme ça, mais je ne
% sais pas si cela paraît évident pour tout le monde.
\[\VAL_I(p \land q) = \begin{cases}
  \true \mbox{ si } \VAL_I(p)= \true  \mbox{ et } \VAL_I(q) = \true \\
  \false\mbox{ sinon}
\end{cases}\]

\[\VAL_I(p \lor q) = \begin{cases}
  \false \mbox{ si } \VAL_I(p)= \false  \mbox{ et } \VAL_I(q) = \false \\
  \true\mbox{ sinon}
\end{cases}\]

\[\VAL_I(\lnot p) = \begin{cases}
  \false\mbox{ si }\VAL_I(p) = \true \\
  \true\mbox{ si }\VAL_I(p) = \false
\end{cases}\]

\[\VAL_I(p \Leftrightarrow q) = \begin{cases}
  \true\mbox{ si }\VAL_I(p) = \VAL_I(q) \\
  \false\mbox{ sinon}
\end{cases}\]

\[\VAL_I(p \Rightarrow q) = \begin{cases}
  \false \mbox{ si } \VAL_I(q) = \false \mbox{ alors que } \VAL_I(p) = \true \\
  \true\mbox{ sinon}
\end{cases}\]

Prenons un exemple concret, utilisons une interprétation pour étudier la phrase
suivante~: « S'il fait beau à midi, j'irai promener le chien ». Nous devons
d’abord traduire cette phrase en l’une des propositions du formalisme que nous
avons défini, en commençant par identifier les propositions premières~:

\begin{enumerate}
\item B = « Il fait beau »~;
\item M = « Il est midi »~;
\item P = « Je vais promener le chien »~;
\end{enumerate}

Identifions également les connecteurs à employer~: « s’il fait beau $\land$
  qu'il est midi $\Rightarrow$ j’irai promener le chien ». En combinant ces deux
résultats, nous obtenons la phrase propositionnelle $(B \land M) \Rightarrow P$.

Interprétons désormais notre proposition à l’aide de l’interprétation suivante~:
\begin{enumerate}
  \item Il fait beau~: $\val_I(B) = \true$~;
  \item Il est midi~: $\val_I(M) = \true$~;
  \item Je n’irai pas promener le chien~: $\val_I(P) = \false$.
\end{enumerate}

Nous pouvons alors effectuer le développement suivant:

\[
VAL_I((B \land M) \implies P) = 
	\begin{cases}
		false \text{ si }VAL_I(P) = false \text{ alors que } VAL_I(B \land M) = true\\
		true  \text{ sinon }
	\end{cases}
\]

\[
VAL_I(B \land M) = 
	\begin{cases}
		true \text{ si }VAL_I(B) = true \text{ et } VAL_I(M) = true\\
		false  \text{ sinon }
	\end{cases}
\]
Dans notre cas: 
\[val_I(B \land M) = true \land true =true \]
Donc: 
\[val_I((B \land M)\implies P) = true \implies \false = false\]


Nous pouvons donc en conclure que, dans cette interprétation, la personne ayant
fait cette affirmation a menti. Notons néanmoins que notre homme n'aurait pas
menti en partant promener le chien alors qu'il pleuvait à midi, rien n'ayant
été dit sur ce qu'il ferait dans le cas où il ne ferait pas beau.

% TODO: Rajouter un exemple simple pour vérifier que l’un des connecteurs est
% bien défini (\land, etc.) et aussi montrer la démarche :
%   - avoir une phrase en français avec 2-3 connecteurs logiques ;
%   - identifier les propositions primaires et les connecteurs ;
%   - réécrire la phrase mathématiquement ;
%   - définir une interprétation ;
%   - l’utiliser pour évaluer la fonction.

\section{Les modèles logiques}

Dans le premier chapitre, nous avons parlés de modèles théoriques (ou théories)
d'un système dans le monde réel.
Dans le contexte de la méthode scientifique,
nous avons fait de la déduction à partir de ces modèles théoriques.
Maintenant que nous avons introduit notre première logique et sa sémantique,
nous pouvons rendre ces notions plus concrètes.

À partir de la notion d’interprétation, nous pouvons définir ce qu’est
un \textit{modèle}. Soit $B = \{b_1, b_2, \dots, b_n \}$ un ensemble de
propositions logiques. Une interprétation $I$ est un modèle de $B$ si et
seulement si $\forall b_i \in B. \VAL_I(b_i) = \true$. Autrement dit, $I$
décrit un univers qui respecte toutes les règles se trouvant dans l’ensemble
$B$.

Dans l'exemple de la section précédente, l’interprétation choisie n’est donc pas un modèle
de la proposition analysée, celle-ci n’étant pas validée. Par contre, 
l’interprétation telle que $\val_I(B) = \true$, $\val_I(M) = \true$ et
$\val_I(P) = \true$ est bien un modèle de la proposition utilisée comme
exemple. Remarquez aussi que cela ne change rien au fait que l’interprétation
choisie soit le modèle d’autres propositions que celle étudiée ou non (par exemple $B
\land M$).

% TODO: Exemple d’expression avec une interprétation qui est un modèle et une
% autre qui n’en est pas un.
\begin{description}\item[Les tautologies :] Pour certaines propositions, toute interprétation est un modèle, c’est-à-dire que ces propositions sont toujours
  vraies. Par exemple, $\true$ est évidemment une tautologie, de même que $A
  \Rightarrow A$ ou encore $A \lor \lnot A$. Le fait qu’une proposition $p$ est
  une tautologie se note $\vDash p$.
\item[Les contradictions :] Pour d’autres propositions, il n’existe aucun
  modèle, c’est-à-dire qu’elles sont toujours fausses. Par exemple $A \land
  \lnot A$ est une contradiction. Le fait qu’une proposition $p$ est une
  contradiction se note $\nvDash p$.
\item[Les contingences :] Toutes les autres propositions sont des contingences. Il
  existe des interprétations qui sont des modèles et d’autres qui n’en sont
  pas. Par exemple $A \land B$ est vrai pour l’interprétation $I$ telle que
  $\val_I(A) = \val_I(B) = \true$, mais faux dans tous les autres cas.
\end{description}

		\subsection{Conséquence logique}
			$p$ est conséquence logique de $q$ si et seulement si $p \Rightarrow q$ est une tautologie. En d'autres termes, si
			\begin{center}
			\begin{tabular}{ll}
			$p \models q$ & $q$ est valide dans tous les modèles de $p$ \\
			&\\
			alors & \\
			$\models (p \Rightarrow q)$ & $p \Rightarrow q$ est une tautologie.\\
			&\\
			On peut donc écrire & \\
			$p \Rrightarrow q$ & $p$ est conséquence logique de $q$.\\
			\end{tabular}
			\end{center}
			Cependant, la conséquence logique ($\Rrightarrow$) n'est pas une proposition logique, mais fait partie du métalangage (cf. syntaxe d'une proposition).
		
		\subsection{Équivalence logique}
			Par le raisonnement ci-dessus, on peut dire que $p$ est logiquement équivalent à $q$ si et seulement si
			\begin{center}
			\begin{tabular}{ll}
			$p \models q$ & $q$ est valide dans tous les modèles de $p$ \\
			$q \models p$ & $p$ est valide dans tous les modèles de $q$ \\
			&\\
			et donc & \\
			$\models (p \Rightarrow q)$ & $p \Rightarrow q$ est une tautologie et\\
			$\models (q \Rightarrow p)$ & $q \Rightarrow p$ est une tautologie.\\
			&\\
			On peut donc écrire & \\
			%impossible de trouver l'équivalence logique en symbole
			$p \Lleftarrow \Rrightarrow q$ & $p$ sont logiquement équivalents $q$.\\ 
			\end{tabular}
			\end{center}
			L'équivalence logique n'est pas non plus une proposition logique.\\
			
			Il ne faut pas non plus oublier la différence entre phrase propositionnelle ($p$, $q$, $s$,...) et propositions premières ($P$, $Q$, $S$,...) (cf. syntaxe d'une proposition) :
			\begin{center}
			\begin{tabular}{lll}
				& $p \Rightarrow q$ & n'est pas une proposition\\
				
				mais &&\\
				&$P \land Q \Rightarrow R \land \lnot S$ & en est bien une.\\
			\end{tabular}
			\end{center}
	
