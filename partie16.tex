\section{Exemples de programmes Prolog}

Maintenant que nous avons vu l'algorithm d'exécution de Prolog,
regardons quelques programmes pour voir comment on peut programmer en Prolog.
L'idée de base est qu'un programme est un ensemble d'axiomes qui énoncent des propriétés vraies
des algorithmes que l'on veut programmer, tout en permettant l'exécution de procéder de façon efficace.

L'invention de Prolog a donné naissance à un nouveau style de programmation.
La programmation de Prolog est l'art de faire une spécification logique qui possède en même temps une exécution efficace.
Dans cette section nous ne pouvons montrer qu'une toute petite partie de ce style,
mais vous êtes encouragés à explorer d'autres programmes en Prolog et à télécharger
un des nombreux systèmes Prolog pour les exécuter.

\subsection{Exemple 1: Factorielle}

Ce premier exemple montre comment on peut faire des calculs avec des nombres en Prolog.

\subsubsection{Programme en Prolog}

Ce programme permet de calculer la factorielle d'un nombre.
La définition est un ensemble de clauses exprimant des propriétés de la fonction factorielle.
Le code commence par un fait: 0!==1.
Ensuite, il définit une clause pour les factorielles de façon généralisée.
Attention, les virgules et les points sont importants.
Les virgules sont les séparateurs entre les littéraux dits négatifs tandis que le point marque la fermeture de la clause.
\begin{verbatim}
fact(0,1).
fact(N,F) :- N>0, 
        N1 is N-1,
        fact(N1,F1), 
        F is N* F1 .
\end{verbatim}

\subsubsection{Requête}

Nous allons maintenant exécuter le programme fact afin de trouver la factorielle de 5 et stocker la réponse dans la variable F.
Le prompt Prolog est représenté par ``\verb+|?-+'' et la sortie standard par ``\verb+->+''.
En Prolog on peut mettre des commentaires après les ``\verb+%+'' ou entre ``\verb+/* */+''.
Il ne faut pas oublier le point à la fin de la requête.
\begin{verbatim} 
|?- fact(5,F). % Requête de l'utilisateur
-> F=120       % Réponse du système
\end{verbatim}

\subsubsection{Forme clausale}
Voici le programme écrit en forme clausale:
\begin{equation}
\begin{array}{l}
\mathrm{fact}(0,1) \\
\wedge \\
(\forall n) (\forall f) (\forall n_1) (\forall f_1) \\
\quad \quad \mathrm{fact}(n,f) \vee \neg(n>0) \vee \neg \mathrm{minus}(n_1,n,1) \vee \\
\quad \quad \neg \mathrm{fact}(n_1,f_1) \vee \neg \mathrm{times}(f,n,f_1)
\end{array}
\end{equation}
Les trois prédicats mathématiques font partie du système et sont prédéfinis:
$a>b$ est vrai si $a>b$,
$\mathrm{minus}(a,b,c)$ est vrai si $a=b-c$,
$\mathrm{times}(a,b,c)$ est vrai si $a=b \times c$.

\subsubsection{Exécution} 

Le but de l'exécution est de prouver la requête $G$ avec $G = \mathrm{fact}(5, r)$.
Pour y arriver, Prolog construira au fur et à mesure de l'exécution
une substitution finale $\sigma_{res}$ en faisant les résolutions qui feront évoluer $r$.
La substitution finale, appliquée à $G$, donnera le résultat final.

\begin{equation}
\begin{array}{l}
\mathit{Requête\ initiale:}\\
r= < \mathrm{fact}(5, r) > \\
\mathit{Résolution\ 1:} \\
\sigma = \{ (n, 5), (f, r) \} \\
r= < (5>0), \mathrm{minus}(n_1, 5, 1), \mathrm{fact}(n_1, f_1), \mathrm{times}(r, 5, f_1) > \\
\mathit{Résolution\ 2:} \\
\sigma'=\sigma \\
r= < \mathrm{minus}(n_1, 5, 1), \mathrm{fact}(n_1, f_1), \mathrm{times}(r, 5, f_1) > \\
\mathit{Résolution\ 3:} \\
\sigma''= \sigma' \cup \{ (n_1, 4) \} \\
r= < \mathrm{fact}(4, f_1), \mathrm{times}(r, 5, f_1) > \\
\ldots \\
\sigma_{res} = \{(r,120), \ldots\} 
\end{array}
\end{equation}

\subsection{Exemple 2: Append de deux listes}

Cet exemple montre comment on peut faire des calculs avec des structures de données en Prolog.
Le but de ce programme est de faire la concaténation de deux listes.

\subsubsection{Programme en Prolog}

\begin{verbatim}
append([],L,L).
append([X|L1],L2,[X|L3]) :- append(L1,L2,L3).
\end{verbatim}

\subsubsection{Requête}

La requête initiale $G = \mathrm{append}(\mathrm{cons}(1,\mathrm{nil}), \mathrm{cons}(2,\mathrm{nil}), l)$.
On veut faire la concaténation des listes \verb+[1]+ et \verb+[2]+ (en syntaxe Prolog).
La paire de liste, écrite \verb+[X|L1]+ en Prolog, s'écrit $\mathrm{cons}(x,l_1)$ en syntaxe logique.

\begin{verbatim}
|?- append([1], [2], L).
-> L=[1,2]
\end{verbatim}

\subsubsection{Forme clausale}

Voici le programme écrit en forme clausale:
\begin{equation}
\begin{array}{l}
\mathrm{append}(\mathrm{nil},l',l') \\
\wedge\\
(\forall l_1) (\forall l_2) (\forall l_3) (\forall x) \\
\quad \quad \mathrm{append}(\mathrm{cons}(x,l_1),l_2,\mathrm{cons}(x,l_3)) \vee \neg\mathrm{append}(l_1,l_2,l_3)
\end{array}
\end{equation}

\subsubsection{Exécution}

La première valeur de $r$ contient la requête initiale.
\begin{equation}
\begin{array}{l}
\mathit{Requête\ initiale:} \\
r = < \mathrm{append}(\mathrm{cons}(1,\mathrm{nil}), \mathrm{cons}(2,\mathrm{nil}), l)> \\
\mathit{Résolution\ 1:} \\
\sigma = \{(x,1), (l_1,\mathrm{nil}), (l_2,\mathrm{cons}(2,\mathrm{nil})), (l,\mathrm{cons}(x,l_3))\} \\
r = < \mathrm{append}(\mathrm{nil}, \mathrm{cons}(2,\mathrm{nil}), l_3) > \\
\mathit{Résolution\ 2:} \\
\sigma_{res} = \sigma \cup \{ (l',\mathrm{cons}(2,\mathrm{nil})), (l_3,l') \} \\
r = < > \\
\end{array}
\end{equation}
Expliquons en plus de détail ce qui se passe lors de résolution 1:
\begin{quote}
Il y a une unification entre le premier élément de $r$,
c'est-à-dire $\mathrm{append}(\mathrm{cons}(1,\mathrm{nil}), \mathrm{cons}(2,\mathrm{nil}), l)$,
et la tête d'une clause (un littéral positif) dans le programme,
c'est-à-dire ici $\mathrm{append}(\mathrm{cons}(x,l_1),l_2,\mathrm{cons}(x,l_3))$.
Cette unification donne lieu à une substitution: une liste de paires (variable,terme)
nécessaire pour que les deux littéraux deviennent égaux.
C'est la substitution $\sigma$.
La nouvelle résolvante après la résolution 1 est le littéral négatif de la clause,
c'est-à-dire $\mathrm{append}(l_1,l_2,l_3)$, où l'on a remplacé les variables par leurs substitutions.
C'est la valeur de $r$ juste en dessous de $\sigma$.
\end{quote}
Après la deuxième résolution,
l'exécution termine avec réussite parce que $r$ est vide.
Le résultat final est:
\begin{equation}
\begin{array}{l}
\sigma_{res} = \{(x,1), (l_1,\mathrm{nil}), (l_2,\mathrm{cons}(2,\mathrm{nil})), (l,\mathrm{cons}(x,l_3)), (l',\mathrm{cons}(2,\mathrm{nil})), (l_3,l') \} \\
G = \mathrm{append}(\mathrm{cons}(1,\mathrm{nil}), \mathrm{cons}(2,\mathrm{nil}), \mathrm{cons}(1,\mathrm{cons}(2,\mathrm{nil})))
\end{array}
\end{equation}
Pour trouver le résultat $G$ on a simplement appliqué $\sigma_{res}$ à la requête initiale.
La seule variable dans la requête initiale est $l$: la valeur de $l$ est à trouver dans $\sigma_{res}$
(il faut suivre les substitutions jusqu'au bout!).

\begin{center}
\noindent\fbox{
\parbox{0.8\textwidth}{Attention, chaque fois que l'on fait la résolution
avec une clause du programme il faut {\em renommer} les variables de la clause.
Dans le cas ci-haut ce sont $l_1$, $l_2$, $l_3$ et $x$.
Ceci est nécessaire pour éviter d'utiliser le même nom pour deux variables différentes pendant l'exécution.}}
\end{center}

\subsection{Exemple 3: Append avec plusieurs solutions}

Le programme de l'exemple précédant peut calculer plusieurs solutions,
si plusieurs solutions existent et on demande à l'algorithme de nous les calculer.
Pour l'exemple 3 on va regarder cela de plus près
en donnant comme requête $\mathrm{append}(l_1, l_2, \mathrm{cons}(1,\mathrm{nil}))$.
On voit bien qu'il y a deux solutions:
\begin{itemize}
\item La première solution est $l_1=\mathrm{nil}$ et $l_2=\mathrm{cons}(1,\mathrm{nil})$.
\item La deuxième solution est $l_1=\mathrm{cons}(1,\mathrm{nil})$ et $l_2=\mathrm{nil}$.
\end{itemize}
Regardons si l'algorithme les trouve aussi!

\subsubsection{Requête}

La requête initiale est $G = \mathrm{append}(l_1, l_2, \mathrm{cons}(1,\mathrm{nil}))$.
On la donne au système Prolog syntaxe \verb+append(L1,L2,[1])+.
L'exécution du système donnera d'abord la première solution.
L'utilisateur pourra demander une deuxième solution en tapant \verb+;+.
\begin{verbatim} 
|?- append (L1,L2,[1]).
-> L1=[], L2=[1] ;
-> L1=[1], L2=[]
\end{verbatim}

\subsubsection{Exécution}

La première valeur de $r$ contient la requête initiale.
\begin{equation}
\begin{array}{l}
\mathit{Requête\ initiale:} \\
r = < \mathrm{append}(l_1, l_2, \mathrm{cons}(1,\mathrm{nil})) > \\
\mathit{Résolution\ 1:} \\
\sigma = \{ (l_1,\mathrm{nil}), (l_2, l'), (l',\mathrm{cons}(1,\mathrm{nil})) \} \\
r =  < >
\end{array}
\end{equation}
L'algorithme commence par choisir la première clause, $\mathrm{append}(\mathrm{nil},l',l')$.
Cela donne un résultat (obtenu en remplissant la requête avec $\sigma$):
\begin{equation}
\begin{array}{l}
G = \mathrm{append}(\mathrm{nil}, \mathrm{cons}(1,\mathrm{nil}), \mathrm{cons}(1,\mathrm{nil})) \\
\end{array}
\end{equation}
En syntaxe Prolog, c'est le résultat \verb+append([],[1],[1])+.
Si on demande un autre résultat,
l'algorithme fera un retour en arrière pour faire un autre choix de clause.
Le dernier choix fait par l'algorithme a été fait lors de la première résolution,
où il a choisi la première clause.
On peut re-exécuter cette résolution, en choisissant la deuxième clause.
Ensuite on continue jusqu'à ce que la résolvante est vide:
\begin{equation}
\begin{array}{l}
\mathit{Résolution\ 1bis:} \\
\sigma = \{ (l_1,\mathrm{cons}(x',l_1')), (l_2,l_2'), (x',1), (l_3',\mathrm{nil}) \} \\
r = < \mathrm{append}(\mathrm{cons}(1,l_1'), l_2', \mathrm{nil}) > \\
\mathit{Résolution\ 2bis:} \\
\sigma' = \sigma \cup \{ (l'',\mathrm{cons}(1,l_1'), (l'',l_2') \} \\
r = < \mathrm{append}(l_1', l_2', \mathrm{nil})> \\
\mathit{Résolution\ 3bis:} \\
\sigma'' = \sigma' \cup \{ (l_1',\mathrm{nil}), (l_2',l'''), (l''',\mathrm{nil}) \} \\
r = < >
\end{array}
\end{equation}
Cela donne un autre résultat (obtenu en remplissant la requête avec $\sigma''$):
\begin{equation}
\begin{array}{l}
G = \mathrm{append}(\mathrm{cons}(1,\mathrm{nil}), \mathrm{nil}, \mathrm{cons}(1,\mathrm{nil}))
\end{array}
\end{equation}
En syntaxe Prolog, c'est le résultat \verb+append([1],[],[1])+.
Cela donne deux résultats qui satisfont tous les deux la requête initiale.
On peut essayer de trouver un troisième résultat, en faisant un nouveau retour en arrière
(c'est possible parce que {\em 3bis} a choisi la première clause),
on verra que l'unification échouera: il n'y a que deux résultats pour notre requête.

\subsection{Dernières remarques}

Voici ce qui termine notre introduction à la programmation logique et au langage Prolog.
Nous n'avons expliqué qu'une petite partie de ce que peut faire un système Prolog.
Je voudrais terminer par attirer votre attention sur les
deux manières de percevoir l'exécution d'un programme Prolog:
\begin{enumerate}
\item {\em La vue opérationnelle}: l'exécution est vue comme une séquence de calculs.
On se concentre sur l'exécution de l'algorithme de preuve.
Le résultat d'un calcul est une séquence d'opérations, avec ses problèmes de temps
d'exécution et d'espace mémoire utilisé.
\item {\em La vue logique}: l'exécution est vue comme une preuve en logique des prédicats.
On oublie l'algorithme, on se concentre sur les formules logiques.
Le résultat d'un calcul est un théorème, donc une conséquence logique des axiomes du programme.
\end{enumerate}
Prolog est un des rares langages à offrir ces deux vues sur une même exécution.
Cela donne beaucoup de puissance: on peut raisonner sur l'exactitude de façon (presque) complètement
indépendante de l'efficacité.

\subsubsection{L'équation de Kowalski}

La relation entre la vue logique et la vue opérationnelle
est résumée par la célèbre équation de Kowalski:
\begin{quote}
Algorithm = Logic + Control
\end{quote}
La logique et le contrôle peuvent être vus de façon séparées.
Pour une même définition logique, on peut changer l'efficacité
en changeant le contrôle.

\subsubsection{L'influence de Prolog}

La programmation logique, le langage Prolog et ses successeurs,
gardent une grande influence sur l'informatique.
L'esprit de Prolog est toujours présent
dans beaucoup de domaines de l'informatique :
\begin{itemize}
\item Les bases de données (modèle relationnel et Datalog)
\item La sémantique Web (basée sur un langage logique OWL et un modèle d'exécution) (basée sur un langage logique OWL et un modèle d'exécution) (basée sur un langage logique OWL et un modèle d'exécution) (basée sur un langage logique OWL et un modèle d'exécution)
\item La programmation par contraintes (on remplace les substitutions par des relations quelconques).
\end{itemize}

