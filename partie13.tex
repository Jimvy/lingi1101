%document réalisé par Scott Ivinza, Florent Lejoly, Cédric Vanden Buckle et Guillaume Demaude
\section{Théorie de l'égalité (EG)}
Il existe différents symboles pour représenter l'égalité : 
\begin{itemize}
	\item E(x,y)
	\item $x = y$
	\item $x == y $
\end{itemize}
Nous allons utiliser $'=='$ dans la suite de ce cours pour représenter l'égalité.

\subsection{Axiomes} 
Pour définir ce qu'est une égalité, nous avons d'abord besoin de définir 3 axiomes et 2 schémas d'axiomes:
\begin{enumerate}
\item Réflexivité \\$\forall x, x==x$
\item Symétrie \\$\forall x, \forall y, x==y \Rightarrow y==x$
\item Transitivité \\$\forall x, \forall y, \forall z, (x==y \land y==z) \Rightarrow x==z$
\item Substituabilité dans les fonctions\\$\forall x_{1}, ...,x, .. x_{n}$ ,  $x_{i}==x \Rightarrow f(x_{1}, ...,x_{i}, .. x_{n}) == f(x_{1}, ...,x, .. x_{n})$
\item Substituabilité dans les prédicats\\$\forall x_{1}, ...,x, .. x_{n}$ ,  $x_{i}==x \Rightarrow P(x_{1}, ...,x_{i}, .. x_{n}) == P(x_{1}, ...,x, .. x_{n})$ 
\end{enumerate}
\subsection{Règles d'inférences}
En plus de ces 5 axiomes, il nous faut aussi définir deux règles d'inférences.\\ \\
Substituabilité fonctionnelle 
	$$ \frac{s_{1}==t_{1} \land s_{2}==t_{2} \land ... \land s_{n}==t_{n}}{f(s_{1},s_{2},...,s_{n}) == f(t_{1},t_{2},...,t_{n})}$$ 
	Substituabilité prédicative 
	$$ \frac{s_{1}==t_{1} \land s_{2}==t_{2} \land ... \land s_{n}==t_{n}}{P(s_{1},s_{2},...,s_{n}) == P(t_{1},t_{2},...,t_{n})}$$ 
Grâce aux règles sémantiques de l'égalité, on peut raisonner sur des formules, mais aussi bien sur une interprétation. \\ 
Si $VAL_{I}(t_{1}) ==  VAL_{I}(t_{2})$ alors $VAL_{I}(t_{1} == t_{2}) = true$
\subsection*{Preuve (Métalangage)}
\noindent Soit I, un modèle de EG ($'=='$) pour $t_{1}$ et $t_{2}$. \\
$VAL_{I}(t_{1} == t_{2}) = VAL_{I}(==)(VAL_{I}(t_{1}), VAL_{I}(t_{2}))$\\
On pose $VAL_{I}(==) = E_{I}$ et $VAL_{I}(t_{1}) = e$ et $VAL_{I}(t_{2}) = e$\\
Donc $VAL_{I}(t_{1} == t_{2}) =E_{I}(e,e)$ \\
$\Rightarrow$ preuve en regardant les axiomes \\
$ VAL_{I}(\forall x, x==x)= true$ (réflexivité) \\
On cherche le $\forall x$ dans la sémantique : \\
On sait que pour tout d $\in$ Domaine de I, $VAL_{I}(d==d)=true$, \\
Si $d = e$\\
alors $E_{I}(e,e)=true$
\subsection{Remarque}
La théorie de l'égalité a une utilité très limitée, elle doit être étendue pour pouvoir servir à quelque chose. On appelle une telle théorie un template.
\section{Théorie de l'ordre partiel (OP)}
On ajoute un deuxième symbole dans le langage en plus du symbole d'égalité.
\begin{itemize}
	\item $==$
	\item $\leq $
\end{itemize}

\subsection{Axiomes} 
Aux axiomes et schémas d'axiomes de la théorie de l'égalité, on rajoute de nouveaux axiomes
\begin{enumerate}
\item Réflexivité \\$\forall x$, $x\leq x$
\item Anti-symétrie \\$\forall x$, $\forall y$, $ x\leq y \land y\leq x\Rightarrow y==x$
\item Transitivité \\$\forall x$, $\forall y$, $\forall z$, $(x\leq y \land y\leq z) \Rightarrow x\leq z$
\item Substituabilité à gauche \\$\forall x_{1}$, $\forall x_{2}$, $\forall x$,  $x_{1}==x \Rightarrow x_{1}\leq x_{2} \Leftrightarrow x \leq x_{2}$
\item Substituabilité à droite \\$\forall x_{1}$, $\forall x_{2}$, $\forall x$,  $x_{2}==x \Rightarrow x_{1}\leq x_{2} \Leftrightarrow x_{1} \leq x$
\end{enumerate}
\underline{Théorème :} $\models \forall x, \forall y,  [x==y \Leftrightarrow (x\leq y)\land (y \leq x)] $
\subsection{Preuve}
La preuve va être démontrée en partie en métalangage et en partie en preuve formelle.\\ \\
\begin{align*}
\Leftrightarrow \text{equivant à :} \Leftarrow \land \Rightarrow \\
\Leftarrow \text{: est démontré par l'axiome antisymétrique} \\	
\Rightarrow \models_{op} \forall x, \forall y, [x\leq y \land y \leq x] 
\end{align*}

\underline{preuve formelle :}
\begin{enumerate}
\item $ \forall x, \forall y, x==y \Rightarrow x \leq x \Leftrightarrow y \leq x$ \hfill substituabilité à gauche
\item $ \forall x, \forall y, x==y \Rightarrow x \leq x \Leftrightarrow x \leq y$ \hfill substituabilité à droite
\item $ x==y \Rightarrow x \leq x \Leftrightarrow y \leq x $ \hfill $\forall$ élimination
\item $ x==y \Rightarrow x \leq x \Leftrightarrow x \leq y $ \hfill $\forall$ élimination
\end{enumerate}
\underline{preuve conditionnelle :}
\begin{enumerate}
\setcounter{enumi}{4}
\item $ x==y$ \hfill supposition
\item $ y\leq x$ \hfill modus ponens (1,4)
\item $ x\leq y$ \hfill modus ponens (2,4)
\item $ y\leq x \land  x\leq y$ \hfill conjonction (6,7)
\item $ x==y \Rightarrow x \leq x \Leftrightarrow y\leq x \land  x\leq y$
\item $ \forall x, \forall y, x==y \Rightarrow x \leq x \Leftrightarrow y\leq x \land  x\leq y$ \hfill $\forall$ Introduction
\end{enumerate}
\subsection{Exemples de modèles d'OP}
\begin{itemize}
\item \underline{$I_{1}$} : $D_{I_{1}} =  \mathbb{Z}$ \\
$ val_{I_{1}}(==) = '=' :$ égalité d'entiers \\
$ val_{I_{1}}(\leq) = '\leq' :$ plus petit ou égal pour les entiers\\
cette interprétation va satisfaire tous les entiers.
\item \underline{$I_{2}$} : $D_{I_{2}} =  P(E)$ ensemble des sous-ensembles de E\\
$ val_{I_{2}}(==) = '=' :$ égalité d'ensemble \\
$ val_{I_{2}}(\leq) = '\subseteq' :$ inclusion d'ensemble
\item \underline{$I_{3}$} : $D_{I_{3}} = ALPH^{2} = {(l_{1},l_{2}),..}$ doublons de lettres de l'alphabet\\
$ val_{I_{3}}(==) = $ égalité des paires \\
$ val_{I_{3}}(\leq) =$ suivant ordre lexicographique\\
exemple : $(l_{i},l_{j}) \leq (l_{p},l_{q})$ si $ l_{i} < l_{p}$ ou $ l_{i} = l_{p}$ et $ l_{j} < l_{q}$ ou $ l_{j} = l_{q}$
\item \underline{$I_{3}$} : $D_{I_{4}} =$ ensemble de listes\\
$ val_{I_{4}}(==) = $ = égalité de liste (si elles possèdent les mêmes composants à la même position)\\
$ val_{I_{3}}(\leq) =$ "suffixe de" \\
exemple : $l_{1} $ est suffixe de $l_{2}$ \\
$l_{1} $ = [d, e, f, g, h] et $l_{2} =$ [a, b, c, d, e, f, g, h, i]
\item \underline{$I_{5}$} : Soit D$_{I_{5}}$ l'ensemble des formules en logique des propositions. On commence à utiliser la logique pour parler d'elle-même:\\
VAL$_{I_{5}}$(==) = "$<\equiv>$" $\rightarrow$ L'égalité est synonyme d'équivalence en logique des propositions
VAL$_{I_{5}}$($\leq$) = "$\models$" $\rightarrow$ Le signe d'inclusion entre les ensembles de modèles.
Maintenant qu'on a défini la sémantique de l'ordre partiel en utilisant la notion de modèles, on peut prendre quelques exemples en raisonnant sur la logique :\\
p $\leq_{I_{5}}$ q   \hspace{1.5cm} p $\models_{I_{5}}$ q \hspace{1.5cm} $\models$ p $\Rightarrow$ q
\item \underline{$I_{6}$} : D$_{I_{6}}$ = l'ensemble des unificateurs d'un ensemble S de termes ou de formules
VAL$_{I_{6}}$(==) = égalité entre substitutions \{ (x$_{i}$,t$_{i}$)... \} : un ensemble de paires avec une variable et un terme VAL$_{I_{6}}$($\leq$) = "moins général que"\\
\newline
Exemple: P(x,f(y)) \hspace{1cm} $\underbrace{P(x,z)}_{\sigma}$\\
$\sigma$' = $\sigma$ $\cup$ \{(z, f(y)\} donc 
$\sigma$' $\leq$ $\sigma$\\
\end{itemize}
Ces exemples montrent qu'on peut raisonner sur tout, y compris sur la logique et les algorithmes eux-mêmes, à partir du moment où on respecte les axiomes.
