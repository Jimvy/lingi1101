\chapter*{Introduction au cours LINGI1101}

Le cours "\textit{Logique et Structures discrètes"} a deux buts importants:

\begin{itemize}
\item Donner la motivation et l'intuition de la logique, pour que cette matière devienne véritablement utile pour les étudiants.
\item Donner les concepts et les formalismes mathématiques nécessaires pour utiliser la logique à bon escient.
\end{itemize}


L'intuition est donc importante pour ce cours, néanmoins, la connaissance des formalismes mathématiques reste essentielle.  Le cours sera coté sur les deux: intuitions (un tiers) et formalismes (deux tiers).\\

\section*{Déroulement du cours}

Le cours est composé de deux parties. La première partie, \textit{logique formelle}, représentera deux tiers du cours. La seconde partie, \textit{structures discrètes sur Internet}, comptera quant à elle pour un tiers du cours.\\

En année académique 2014-2015,
l'évaluation de ce cours se compose de trois parties. Il y aura tout d'abord une interrogation au milieu du quadrimestre portant sur 5 points. Il vous sera également demandé de prendre note pendant une heure de cours par groupe de trois, ceci afin de contribuer au syllabus. Ces notes prises au cours rapporteront au maximum 2 points de la note finale à chacun des participants. L'examen sera divisé en deux parties. La première partie sur 5 points portera sur la matière de l'interrogation. La note retenue sera le maximum entre la note de l'interrogation et celle obtenue à la question de l'examen. La seconde partie de l'examen sera donc cotée sur 13 points et portera sur le reste de la matière.\\

Le cours se base principalement sur trois ouvrages :
\begin{itemize}
\item Introductory Logic and Sets for Computer Scientists, by \textit{Nimal Nissanke}.
\item Mathématiques discrètes : bases logiques de l'informatique, by \textit{Philippe Delsarte and Axel van Lamsweerde}.
\item Networks, Crowds, and Markets: Reasoning About a Highly Connected World, by \textit{David Easley and Jon Kleinberg}. \footnote{Quelques chapitres.}
\end{itemize}
Néanmoins, ce syllabus peut être lu indépendamment de ces ouvrages.
Le syllabus complémente
ces ouvrages par d'autres informations afin que le tout donne une vue équilibrée du domaine.

\section*{Plan du cours}

Cette partie va parler du rôle des raisonnements et des différentes formes de raisonnement. Nous prendrons en exemple la méthode scientifique.

\subsection*{Logique des propositions}

La logique des propositions est un langage formel constitué d'une syntaxe et d'une sémantique. La syntaxe décrit l'ensemble des formules qui appartiennent au langage. La sémantique permet de donner un sens aux formules de langage. C'est une logique très ancienne qui vient de l'antiquité. 

\subsection*{Logique des prédicats}

C'est une logique beaucoup plus expressive et la plupart des travaux mathématiques peuvent être écrits dans ce langage\footnote{Elle est un effet un parfait compromis entre expressivité et efficacité.}. Elle est aussi définie comme la logique du premier ordre.\footnote{Il existe d'autres formes de logiques plus expressives, mais plus difficiles à utiliser. Exemple : la logique du deuxième ordre. } En logique des prédicats, les éléments de base du langage ne sont plus des propositions, mais des prédicats. 

\subsection*{Interprétations et modèles}

La logique a besoin d'un langage, de phrases pour la décrire. Cette section couvrira donc la sémantique à utiliser.

\subsection*{Théorie de la preuve}

Nous pouvons manipuler une phrase en logique pour obtenir un résultat. Par exemple, si A et B sont vrais, nous pouvons en déduire que A est vrai. Il y a des règles d'inférences à utiliser pour prendre une phrase en logique et en déduire une autre.
Une preuve mathématique est une séquence de phrases liées par des règles d'inférences.

\subsection*{Algorithme de preuve}
C'est l'algorithme le plus puissant qui existe en logique des prédicats. Néanmoins, il est inefficace seul. Afin de le rendre efficace, il faut poser des hypothèses. Nous approfondirons ce problème dans le cadre de cette section. 

\subsection*{Théorie logique}

Il est possible de formaliser tout objet mathématique avec une théorie logique qui lui est propre.  En exemple, citons la théorie des ensembles, des fonctions et des ordres partiels.

\subsection*{Programmation logique}

Le rêve serait de pouvoir exprimer toute chose logique en langage de programmation efficace. Il s'agira d'appliquer ce principe avec l'algorithme de preuve, sur base d'hypothèses.

\part{Logique formelle}
\chapter{Contexte: la méthode scientifique}

\section{Formalisation d'un système}

Comment pouvons-nous formaliser un système dans le monde réel tels que les champs magnétiques ou la gravitation ?

\begin{center}
\includegraphics[scale=0.65]{images/Abstraction.pdf}
\end{center}

Afin de formaliser un système dans le monde réel, nous devons faire une abstraction vers un modèle théorique. Ce modèle théorique, aussi appelé théorie, est un ensemble de phrases logiques dont il est possible de tirer des prédictions en utilisant le raisonnement déductif. Il n'est intéressant que s'il se comporte comme le vrai système.

Un exemple de cette formalisation pourrait être les équations de Maxwell qui sont le modèle théorique correspondant pour l'électromagnétisme.

\section{Boucle de raisonnement}

Il existe trois formes de raisonnement: la déduction, l'induction et l'abduction. Ces trois formes de raisonnement peuvent être liées dans une boucle de raisonnement de la façon suivante:

\begin{center}
\includegraphics[scale=0.50]{images/BoucleRaisonnement1.pdf}
\end{center}

\subsection{Déduction}

Il s'agit de faire des calculs et des raisonnements logiques par rapport à une théorie. 
Avec ces raisonnements, on déduit le résultat qu'une expérience donnerait selon la théorie.
Par exemple, en utilisant les équations de Maxwell on peut déduire le trajectoire d'un
objet avec une charge électrique dans un champ électromagnétique.  \\

\subsection{Induction}

L'induction est le fait de trouver une règle générale à partir des expériences répétées.
On choisit en général une règle moyenne qui deviendra la règle générale.  Il faut souligner que les résultats expérimentaux ne sont pas totalement fiables ou complets. Dès lors, la règle trouvée n'est pas nécessairement exacte.  Par exemple, si par induction nous avons trouvé la règle, "les oiseaux volent", cela est vrai tant que l'on n’a pas vu un pingouin. Autre exemple, nous pouvons supposer que demain le soleil va se lever comme depuis des milliers d'années, même si rien ne l'assure.\\

\subsection{Abduction}

On compare la règle générale trouvée lors de l'induction avec la théorie.
S'il y a une incohérence entre la règle générale et la théorie qui ne rentre pas dans la marge d'erreur expérimentale, on suppose qu'il y a une erreur dans la théorie.
Il faut alors corriger la théorie existante ou en inventer/deviner une nouvelle. 
Ce type de raisonnement s'appelle l'abduction:
trouver une {\em explication} (= la théorie corrigée) pour une règle ou un fait.
On applique l'abduction couramment dans la vie de tous les jours; par exemple, lorsqu'un élève entre trempé dans la classe, nous supposons qu'il pleut dehors.
La pluie est une explication possible pour l'état de l'élève. \\

\subsection{Conclusion}

Sur ces trois formes de raisonnement, seule la déduction est un raisonnement sûr.
Les deux autres, l'induction et l'abduction, peuvent donner des erreurs.
Malgré cela, les trois formes sont tout aussi importantes.
Par exemple, il faut les trois pour expliquer comment marche la méthode scientifique.
Dans l'état actuel de la science du raisonnement,
nous comprenons beaucoup mieux la déduction que l'induction et l'abduction.
Toute la logique mathématique est un approfondissement de la science de la déduction.
Nous nous focaliserons dans ce cours uniquement sur la déduction. \\

\section{Exemples}

\subsection{Loi de Maxwell}

Nous illustrons dès à présent le fonctionnement de la boucle de raisonnement à l'aide de l'exemple cité plus haut, c'est-à-dire les équations de Maxwell :

\begin{center}
\includegraphics[scale=0.50]{images/BoucleRaisonnement2.pdf}
\end{center}

Par déduction, grâce à la théorie et aux conditions initiales que nous fixons, nous calculons
la trajectoire d'un électron. Nous effectuons ensuite des mesures dans le monde réel. Nous allons, par exemple, mesurer la trajectoire plusieurs fois avec des méthodes différentes et, par induction, nous trouvons une règle qui est la loi de comportement de la particule.
Nous comparons ensuite cette règle à la théorie, et nous la corrigeons si besoin.
La création d'une théorie qui explique la règle trouvée est une abduction.
\\



\subsection{Sac de billes}

Afin d'illustrer les 3 formes de raisonnements de manière plus formelle, considérons un sac de billes pouvant contenir des billes noires ou blanches. Notons que sac(x) signifie "la bille x est dans le sac" et que blanc(x) signifie "la bille x est blanche". \\

\subsubsection{Déduction}

\begin{enumerate}
  \item Règle: $\forall$ $x$, $sac(x)$ $\Rightarrow$ $blanc(x)$
  \item Cas: $sac(a)$, $sac(b)$, $\cdots$\\
  \rule{5.5cm}{.1pt} 
  \item Résultat: $blanc(a)$, $blanc(b)$, $\cdots$
\end{enumerate}

Si toutes les billes se trouvant dans le sac sont blanches et que l'on pioche une bille de ce sac, cette bille sera blanche. Cette déduction est forcément correcte.

\subsubsection{Induction}

\begin{enumerate}
  \item Cas: $sac(a)$, $sac(b)$,$\cdots$
  \item Résultat: $blanc(a)$, $blanc(b)$, $\cdots$\\
  \rule{5.5cm}{.1pt}	
  \item Règle: $\forall$ $x$, $sac(x)$ $\Rightarrow$ $blanc(x)$
\end{enumerate}

Si toutes les billes que l'on pioche du sac sont blanches, alors nous pouvons établir comme règle que toutes les billes dans le sac sont blanches. Cette induction n'est pas forcément correcte.

\subsubsection{Abduction}

\begin{enumerate}
  \item Règle: $\forall$ $x$, $sac(x)$ $\Rightarrow$ $blanc(x)$
  \item Résultat: $blanc(a)$, $blanc(b)$, $\cdots$\\
  \rule{5.5cm}{.1pt}
  \item Cas: $sac(a)$, $sac(b)$, $\cdots$
\end{enumerate}

Si toutes les billes se trouvant dans le sac sont blanches et que nous trouvons des billes blanches à côté du sac, nous pouvons penser qu'elles viennent du sac.
L'explication pour la couleur des billes trouvées est qu'elles viennent du sac.
Cette abduction n'est pas forcément correcte.
