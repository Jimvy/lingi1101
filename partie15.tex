\section{Introduction à la programmation logique}
\label{prolog}
\subsection{Introduction à la programmation logique}

\textbf{Prolog} est l'un des principaux langages de programmation logique. Il est à la base de nombreux fondements. 

La programmation logique fait de la déduction sur les axiomes.
On utilise la logique comme un langage de programmation : on va adapter l'algorithme de réfutation vu précédemment.

Le programme (qui est similaire à une théorie):
\begin{itemize}
	\item Axiomes en logique des prédicats.
	\item Une requête, un but (le goal) : le but du système est d'apporter une preuve.
	\item Un prouveur de théorème. Attention : il faut des conditions sur le prouveur car il faut être capable de prévoir le temps et l'espace utilisé par le programme.
\end{itemize}

\paragraph{}
Exécuter un programme = faire des déductions en essayant de prouver le but. Mais est-ce que cette idée peut donner un système de programmation pratique?

\paragraph{}
Il y a un compromis entre expressivité et efficacité: si c'est trop expressif, ça devient moins efficace, par contre si c'est trop peu expressif, on ne peut rien programmer, ça ne sert à rien non plus. Il faut donc être expressif tout en restant efficace. Prolog offre un bon équilibre entre expressivité et efficacité.

\paragraph{}
Mais pour arriver à cela, il y a quelques problèmes à surmonter:
\begin{enumerate}\renewcommand {\theenumi }{\alph {enumi}}
	\item un prouveur est limité :\\
		vérité = $ p \models q $ (= $ q $ est vrai dans tous les modèles de $ p $)\\
		preuve = $ p  \vdash q $ \\
		$ p \models q \Rightarrow p \vdash q $ (= Si c’est vrai dans tous les modèles, on peut trouver une preuve)\\
		Si $ p \models q $ alors algorithme se terminera. Cependant, on ne peut trouver de preuve que pour des choses vraies dans tous les modèles (en pratique il est impossible d'analyser l'ensemble des modèles, on ne prend qu'une partie des modèles ce qui limite le programme).
		
		\item Même si l'on peut trouver une preuve, le prouveur est peut-être inefficace (utilise trop de temps ou de mémoire) ou imprévisible. $ \to $ On ne peut pas raisonner sur l'efficacité du prouveur.
		
		\item La déduction faite par le prouveur doit être constructive  
\end{enumerate}

		Si le prouveur affirme : $ (\exists X) P(X) $  alors le prouveur doit donner une valeur de x (c'est quoi $ x $).



Il faut construire un résultat.

\underline{Pour résoudre ces problèmes}
\begin{enumerate}
	\item Restrictions sur la forme des axiomes.\\
	typiquement : 
	$$(\forall X_{1}) ... (\forall X_{n}) A_{1} \wedge A_{2} \wedge ... \wedge A_{n} \Rightarrow A$$
	$$C_{i} = \neg A_{1} \vee \neg A_{2} \vee ... \vee \neg A_{n} \vee A$$
	Il n’y a qu’un seul littéral sans négation. (Pour prouver A, il faut prouver $A_{1}$; $A_{2}$, ..., $A_{n}$).\\
	$C_{i}$ est une clause. Le programme tout entier est une série de clauses :
	$$C_{1} \wedge C_{2} \wedge ... \wedge C_{k}$$
	\item Le programmeur va aider le prouveur.
	Par exemple : il faut commencer par prouver $A_{1}$ puis $A_{2}$... dans cet ordre là.\\
	Le programmeur donnera des heuristiques, des méthodes de calcul. Attention : ces heuristiques ne changent pas la sémantique logique du programme. Elles ont seulement un effet sur l'efficacité !
	$$\text{ordre des axiomes}\left \{
		\begin{array}{l}
		C_{1} = \neg A_{1} \vee \neg A_{2} \vee ... \vee A_{n} \vee A \\
	
		C_{2} = \underbrace{\neg B_{1} \vee \neg B_{2} \vee ... \vee B_{k} \vee A}_{\text{ordre des littéraux dans un axiome}}
		\end{array}
		\right.$$
	
\end{enumerate}

Le langage Prolog utilise ces deux ordres.

\underline{Bref historique :}
\begin{enumerate}
	\item 1965 : 	La règle de résolution a été inventée par A. Robinson
	\item 1972 :	Invention du langage Prolog / premier interprète (de Prolog) par A. Calmerauer, R. Kowalski et Ph. Roussel. 
Ils voulaient faire un langage de programmation logique, et connaissaient les différentes formes de logique ainsi que la résolution. Ils ont donc inventé un langage très simple qu'ils ont appelé Prolog (pour Programmation Logique).  Il s'avère que ce langage a un compromis très intéressant par rapport à la tension entre efficacité et expressivité.
Aujourd'hui, on peut faire une implémentation extrêmement efficace de Prolog. Il est extrêmement expressif, ce qui permet de faire des programmes complexes. C'est un langage à part entière.
\end{enumerate}

\subsection{Introduction à Prolog}

En Prolog, on a des clauses (règles):
	\begin{equation}
		 A1 \leftarrow B1, … , Bn
	\end{equation}
	 (On peut prouver $ A $ en prouvant $ B1 $ jusqu’à $ Bn $. Remarque :$ \leftarrow $ou :- )
	 
	 \begin{equation}
	 	\neg (B_{1} \wedge ... \wedge B_{n}) \vee A_{1}
	 \end{equation}
	 \begin{equation}
	 	 (\neg B_{1} \vee \neg B_{2} \vee ... \vee \neg B_{n} \vee A_{1})
	 \end{equation}
	 
	 Programme = ensemble de clauses.
	 
	 \underline{Exemple d’un petit programme en Prolog:} (extrait du livre « The Art of Prolog » par L. Sterling et E. Shapine)
	 
	 Règle : $ grandpere(x, z) \leftarrow pere(x,y), pere(y,z) $. ($x$, $y$,… sont des variables. En Prolog, elles sont souvent en majuscule)
	 
	 \begin{table}[h]
	 	\begin{tabular}{ll}
	 		Faits: & $pere(terach, abraham)$ (terach, abraham,… sont des constantes) \\
	 		& $pere(abraham, isaac)$\\
	 		& $pere(haram, lot)$ \\
	 		& $…$ \\\\
	 		$ \to $ & Syntaxe clausale: $pere(terach,abraham) \wedge pere(abraham,isaac) \wedge  $ \\
	 		& $ pere(haram,lot) \wedge (\neg pere(x,y) \vee \neg pere(y,z) \vee grandpere(x,z))      $                                                         
	 	\end{tabular}
	 \end{table}
	 
	Il existe une corrélation évidente entre Prolog et les bases de données. Elles ont été inventées quasi au même moment et aujourd'hui Prolog est utilisé comme une sémantique pour les bases de données déductibles. Ici, on peut avoir une relation avec deux colonnes qui auraient l'argument père. le grand-père serait une combinaison de ces deux relations.
	 
	 Prolog peut être vu comme une sorte de base de données relationnelle mais beaucoup plus puissante : on peut faire des programmes qui sont plus que des simples requêtes, avec des calculs beaucoup plus complexes.
	 
\subsection{Algorithme d'exécution de Prolog}

Dans la version de l’algorithme de preuve par résolution, l’ensemble S grandit (ce qui n’est pas très efficace).

L’idée :
\begin{itemize}
	\item On commence par mettre le but (G) que l’on veut prouver dans r (sans négation).
	\item Ensuite, jusqu’à ce que r soit vide, on prend le premier littéral dans $r(A_{1})$.
	\item Puis, on parcourt un à un les axiomes de P (P est le programme, la base de faits) pour trouver une clause $Ax_{i}$ unifiable avec $A_{1}$ au moyen de $\sigma(u.p.g)$
	\begin{itemize}
		\item Si on trouve une telle clause, on ajoute à r les littéraux de $Ax_{i}$ après unification avec $A_{1}$ (et on recommence au début).
		\item Si on ne trouve pas de clause unifiable, on revient sur le dernier choix (Par exemple, pour un $A_{1}$ qui aurait plusieurs clauses unifiables, on a dû en choisir une. Et bien, on retourne en arrière pour en choisir une autre, sans oublier de modifier r. (En effet, il faut éliminer les résultats de toutes les unifications qui ont été réalisées entre le moment où le point de choix a été mémorisé et le moment du retour en arrière.))
		\item Si on épuise tous les choix sans que r soit vide alors nous sommes en cas d’échec.
	\end{itemize}
	\item Lorsque le programme s’arrête, si r est vide, on a un résultat.
\end{itemize}

Programme  : $P={Ax_{1}, ..., Ax_{n}}$

Un « but » (un goal, une « requête ») G ($\simeq$ théorie)


\begin{algorithm}[H]
$r := <G>$ … résolvante (une séquence de littéraux $\rightarrow$ $r= <A_{1},A_{2},...,A_{m}>$  Il n’y a qu’un seul r.)\\
\While{r est non vide}{
	\begin{itemize}
		\item Choisir un littéral $A_{1}$ dans r (on prend le premier littéral)
		\item Choisir une clause $Ax_{1}=(A \leftarrow B_{1},...,B_{k})$ dans P. (d’abord on prend la première clause, puis la suivante jusqu’à ce qu’on trouve une clause unifiable avec $A_{1}$. Si aucune clause n’est unifiable on revient sur le dernier choix (backtrack))
		\item Nouvelle résolvante $:= <B_{1},..., B_{k}, A_{2},...,A_{m}> \sigma$
		\item $G^{'} = G \sigma$
	\end{itemize}
}
\If{r est vide}{le résultat est le dernier $G^{'}$}
\If{on épuise les choix sans que r soit vide}{le résultat est NON. ( On n’a pas prouvé G). (Attention : G est peut-être vrai, mais les heuristiques ne suffisent pas pour le prouver.)}
\end{algorithm}

On peut également avoir une boucle infinie (l’algorithme est non déterministe).
