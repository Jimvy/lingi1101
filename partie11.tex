\chapter{Théorie logique}
\section{Étude des structures discrètes}
\noindent
\textbf{Exemples de structures discrètes} : entiers positifs, chaînes, arbres, ensembles, relations, fonctions ...\\

On peut définir ces structures avec la logique des prédicats et faire des raisonnements sur celles-ci en utilisant des règles d'inférence.\\

\noindent
\textbf{Utilisation} :\begin{itemize}
\item On peut programmer avec ces structures. \textit{Prolog} permet d'écrire les axiomes directement. Il faut cependant faire attention de bien les choisir ;
\item On peut les utiliser dans les assistants de preuve (exemples d'assistants de preuve : \textit{Coq, Isabelle})\\
\end{itemize}

\section{Théorie du premier ordre}
\noindent
\textbf{Intuition} : définition logique d'une structure mathématique

\noindent
\subsection{Définition d'une théorie}
\begin{itemize}
\item Sous-langage de la logique du premier ordre
\begin{itemize}
\item \textbf{vocabulaire} : constantes, fonctions, prédicats ;
\item règles syntaxiques et sémantiques sur ce vocabulaire ;
\end{itemize}
\item Ensemble d'axiomes (formules fermées, c'est-à-dire formules ne contenant pas de variables libres);
\item Ensemble de règles d'inférence.
\end{itemize}

\subsection{Exemple : théorie des liens familiaux (\bsc{fam})}
\begin{enumerate}
\item Vocabulaire : 
\begin{itemize}
\item 2 symboles de fonctions : $p/1$, $m/1$
\item 3 symboles de prédicats : $P/2$, $GM/2$, $GP/2$\\
\end{itemize}

On peut interpréter les fonctions $p$ et $m$ comme "père de" et "mère de", et les prédicats $P$, $GM$ et $GP$ comme "parent de", "grand-mère de" et "grand-père de". On donnera plus de précisions sur cette interprétation par la suite.\\

\item Axiomes : \\
\begin{center}
\begin{tabular}{lcc}
$(\forall x) \left(P(x,p(x))\right)$ & \hspace*{2cm}& (père)\\
$(\forall x) \left(P(x,m(x))\right)$ & \hspace*{2cm}& (mère)\\
$(\forall x)(\forall y) \left(P(x,y)\Rightarrow GP(x,p(y)) \right)$&&\\
$(\forall x)(\forall y) \left(P(x,y)\Rightarrow GM(x,m(y)) \right)$&&\\
\end{tabular}
\end{center}

\item Règles : \\
Les règles sont uniquement celles de la logique des prédicats.
\end{enumerate}
\subsubsection{Première interprétation}
\begin{tabular}{lll}
$D_I$ : personnes&&\\
$\text{val}_I(p)=\text{"père de"}$ &\hspace*{1cm} &$\text{père de} : \text{Pers}\rightarrow\text{Pers} : d \rightarrow \text{"père de" }d$\\
$\text{val}_I(m)=\text{"mère de"}$ &\hspace*{1cm} &$\text{mère de} : \text{Pers}\rightarrow\text{Pers} : d \rightarrow \text{"mère de" }d$\\
$\text{val}_I(P)=\text{"Parent"}$ &\hspace*{1cm} &$\text{Parent}(d_1,d_2)=T$ ssi $d_2$ est un parent de $d_1$\\
$\text{val}_I(GP)=\text{"Grand-père"}$ &\hspace*{1cm} &$\text{Grand-père}(d_1,d_2)=T$ ssi $d_2$ est un grand-père de $d_1$\\
$\text{val}_I(GM)=\text{"Grand-mère"}$ &\hspace*{1cm} &$\text{Grand-mère}(d_1,d_2)=T$ ssi $d_2$ est une grand-mère de $d_1$\\
\end{tabular}\\

Cette interprétation est un modèle de \bsc{fam} car les axiomes sont tous vérifiés.\\
On remarque la ressemblance avec une théorie scientique, où les axiomes correspondent à la théorie et l'interprétation à ce que celle-ci signifie dans le monde réel.\\
Une théorie peut avoir plusieurs modèles.\\ 
\subsubsection{Seconde interprétation}
Cette interprétation est également un modèle de \bsc{fam}.\\

\begin{tabular}{lll}
$D_J$ : $\mathbb{N}$&&\\
$\text{val}_J(p)="p_J"$ &\hspace*{1cm} &$p_J : \mathbb{N}\rightarrow\mathbb{N} : d \rightarrow 2d$\\
$\text{val}_J(m)="m_J"$ &\hspace*{1cm} &$m_J : \mathbb{N}\rightarrow\mathbb{N} : d \rightarrow 3d$\\
$\text{val}_J(P)="P_J"$ &\hspace*{1cm} &$P_J(d_1,d_2)$ ssi $d_2=2d_1$ ou $d_2=3d_1$\\
$\text{val}_J(GP)="GP_J"$ &\hspace*{1cm} &$GP_J(d_1,d_2)$ ssi $d_2=4d_1$ ou $d_2=6d_1$\\
$\text{val}_J(GM)="GM_J"$ &\hspace*{1cm} &$GM_J(d_1,d_2)$ ssi $d_2=6d_1$ ou $d_2=9d_1$\\
\end{tabular}\\

\section{Propriétés des théories}
\begin{itemize}
\item Une formule fermée $p$ est \textbf{valide} dans la théorie $Th$ si elle est vraie dans chaque modèle de $Th$. On écrit : 
$$\vDash_{Th} p$$
Soit l'ensemble des axiomes $Ax=\{Ax_1, \hdots, Ax_n\}$. On a bien que $\vDash_{Th} Ax_i$.\\
\item $q$ est une \textbf{conséquence logique} de $p$ dans la théorie $Th$ si $q$ est vraie dans tous les modèles de $Th$ qui rendent $p$ vraie. On écrit :  
$$p \vDash_{Th} q$$
\item Une théorie est \textbf{consistante} si elle a au moins un modèle ($>0$ modèles).
\item Une théorie est \textbf{inconsistante} si elle n'a pas de modèle ($0$ modèles).\\
\end{itemize}

Comment peut-on faire pour établir $\vDash_{Th} p$? Il y a deux approches différentes :
\begin{enumerate}
\item \textbf{l'approche sémantique} : on prend un modèle quelconque de $Th$ et on évalue $\text{VAL}_I (p)$ en utilisant le fait que $\text{VAL}_I (Ax_i)=T$ ;
\item \textbf{l'approche syntaxique} : théorie de preuve : on essaye de construire une preuve de $p$ à partir des axiomes, en appliquant les règles de $Th$.\\
\end{enumerate}
En pratique, la deuxième approche est beaucoup plus souvent utilisée.\\


\subsubsection{Preuve (esquisse)}
La preuve qui suit est une esquisse. Il manque plusieurs étapes.\\
On veut montrer  : 
$$\vDash_{\text{\bsc{fam}}} (\forall x)(\exists z) GM(x,z)$$

\begin{tabular}{lll}
$(\forall x)(\forall y) \left(P(x,y)\Rightarrow GM(x, m(y)) \right)$&\hspace*{1cm}&($Ax$)\\
$(\forall x) \left(P(x,p(x))\Rightarrow GM(x, m(p(x))) \right)$&\hspace*{1cm}&(Elimination de $\forall y$ et substitution $y/p(x)$)\\
$\left( \forall x \hspace*{0.1cm} P(x,p(x)) \right)\Rightarrow \forall x \hspace*{0.1cm} GM(x,m(p(x)))$&\hspace*{1cm}&(Distribution $\forall/\Rightarrow$)\\
$\forall x \hspace*{0.1cm} GM(x,m(p(x)))$&\hspace*{1cm}&(Modus ponens)\\
$\forall x \hspace*{0.1cm} \exists y \hspace*{0.1cm} GM(x,y)$&\hspace*{1cm}&(Introduction de $\exists$)\\
\end{tabular}\\

\section{Qualité d'une théorie}
Certaines qualités sont directement issues de la logique:
\begin {enumerate}
\item {\textbf{consistante}} : il est impossible de déduire \(p\) et \(\neg p\)  de la même théorie.
\item {\textbf{minimale}} : les axiomes sont indépendants: 
$\{Ax_1, \hdots, Ax_n\}$   $\nvDash Ax_k $\\
\textbf{Vérification} : Construction d'interprétations.\\
$\text{VAL}_J (Ax_k)= False$\\
$\text{VAL}_J (Ax_i)= True$  
pour $i \neq k$
\item {\textbf{complète}} : les axiomes suffisent pour prouver la propriété d'intérêt. Sinon il faut en ajouter.
\end {enumerate}
\subsection{Exemple : qualité de deux théories}
\begin {enumerate}
\item {\textbf{Système de Copernic}} : Chaque axiome de chaque planète est indépendant, car elles tournent toutes autour du soleil.
\item{\textbf{Système de Ptolémée}} : Les axiomes de chaque planète dépendent de ceux de la Terre, car elle représente le centre de l'univers, mais elle est également une planète et dispose donc d'axiomes.
\end {enumerate}
