\chapter{Théories logiques}

On peut utiliser la logique des prédicats pour définir et étudier des structures mathématiques.
L'avantage est qu'une définition avec la logique est complètement précise.
La définition d'une structure contiendra deux types de formules:
des {\em axiomes} et des {\em règles d'inférence}.
Quelques exemples simples des structures que l'on peut définir sont
l'ordre partiel, les treillis, les groupes, les entiers, les chaînes, les arbres,
les ensembles, les relations et les fonctions.
Il y en a beaucoup plus que cela; en fait quasi toutes les mathématiques
peuvent êtres formalisées avec la logique des prédicats.
Dans ce chapitre nous allons voir quelques exemples de ces formalisations
pour des structures discrètes.

Ces formalisations peuvent être utilisées pour la programmation (par ex. avec Prolog)
et aussi dans les assistants de preuve.
Pour ces deux utilisations, il faut faire très attention à la manière dont on définit
les axiomes et les règles: cela peut avoir une grande influence sur l'efficacité de
l'exécution.
Mais dans ce chapitre nous ne nous intéressons pas à l'efficacité, mais plutôt à
la simplicité et la compréhension.

\subsubsection{Logique du premier ordre}

La logique des prédicats est parfois appelé la logique du premier ordre.
Dans ce chapitre nous allons parfois utiliser cette terminologie.
On l'appelle premier ordre parce que le domaine des quantificateurs $\forall x$ et $\exists x$
est $D_I$: les valeurs de $\mathrm{val}_I(x)$ sont toujours dans $D_I$.
Il existe aussi des logiques d'ordres superieurs; dans ce cas le domaine des quantificateurs
peut être autre chose.
Par exemple, dans une logique du second ordre, on peut quantifier sur les prédicats et pas
simplement sur les éléments du domaine de discours.
On pourra donc dire $\forall p$ ou $\exists p$ où $p$ est un prédicat.
Cela donne une expressivité supplémentaire par rapport à la logique du premier ordre,
mais les raisonnements et les preuves deviennent beaucoup plus complexes.
Il y a aussi moins de résultats généraux sur cette logique
et la plupart des mathématiques utilisées en pratique n'ont pas besoin de cette logique
pour être formalisée.
Nous n'abordons pas les logiques d'ordres supérieurs dans ce cours.

\section{Théorie du premier ordre}

Le concept de base est la {\em théorie du premier ordre}.
Une théorie du premier ordre est un ensemble $B$ de formules
qui représente les axiomes et les règles d'inférence d'une structure mathématique.
Nous allons nous restreindre au modèles: les interprétations de $B$
où les axiomes et les règles sont valides.

\subsection{Définition d'une théorie}

Une théorie contient les partie suivantes:
\begin{itemize}
\item Un sous-langage de la logique du premier ordre:
\begin{itemize}
\item Vocabulaire : constantes, fonctions, prédicats ;
\item Règles syntaxiques et sémantiques sur ce vocabulaire ;
\end{itemize}
\item Ensemble d'axiomes
\item Ensemble de règles d'inférence
\end{itemize}
Les axiomes et les règles d'inférence sont des
formules {\em fermées}, c'est-à-dire des formules ne contenant pas de variables libres.
L'idée est que ces formules expriment des faits sur tous les éléments du domaine.

\subsection{Exemple : théorie des liens familiaux (\bsc{FAM})}

Comme premier exemple
nous définissons une théorie qui permettra de raisonner sur les liens familiaux.

\begin{enumerate}
\item Vocabulaire : 
\begin{itemize}
\item 2 symboles de fonctions à un paramètre : $p/1$, $m/1$
\item 3 symboles de prédicats à deux paramètres : $P/2$, $GM/2$, $GP/2$\\
\end{itemize}
On peut interpréter les fonctions $p$ et $m$ comme "père de" et "mère de", et les prédicats $P$, $GM$ et $GP$ comme "parent de", "grand-mère de" et "grand-père de". On donnera plus de précisions sur cette interprétation par la suite.\\

\item Axiomes :
\begin{center}
\begin{tabular}{lcr}
$(\forall x) \left(P(x,p(x))\right)$ & \hspace*{2cm}& (père)\\
$(\forall x) \left(P(x,m(x))\right)$ & \hspace*{2cm}& (mère)\\
$(\forall x)(\forall y) \left(P(x,y)\Rightarrow GP(x,p(y)) \right)$&& (grand-père)\\
$(\forall x)(\forall y) \left(P(x,y)\Rightarrow GM(x,m(y)) \right)$&& (grand-mère)\\
\end{tabular}
\end{center}
\vspace{\baselineskip}

\item Règles : \\
Les règles sont uniquement celles de la logique des prédicats.
\end{enumerate}

\subsubsection{Première interprétation}
\begin{tabular}{@{}llr}
$D_I$ : personnes&&\\
$\text{val}_I(p)=\text{"père de"}$ & &$\text{père de} : \text{Pers}\rightarrow\text{Pers} : d \rightarrow \text{"père de" }d$\\
$\text{val}_I(m)=\text{"mère de"}$ & &$\text{mère de} : \text{Pers}\rightarrow\text{Pers} : d \rightarrow \text{"mère de" }d$\\
$\text{val}_I(P)=\text{"Parent"}$ & &$\text{Parent}(d_1,d_2)=T$ ssi $d_2$ est un parent de $d_1$\\
$\text{val}_I(GP)=\text{"Grand-père"}$ & &$\text{Grand-père}(d_1,d_2)=T$ ssi $d_2$ est un grand-père de $d_1$\\
$\text{val}_I(GM)=\text{"Grand-mère"}$ & &$\text{Grand-mère}(d_1,d_2)=T$ ssi $d_2$ est une grand-mère de $d_1$\\
\end{tabular}\\

Cette interprétation est un modèle de \bsc{fam} car les axiomes sont tous vérifiés.
On remarque la ressemblance avec une théorie scientique, où les axiomes correspondent à la théorie
et l'interprétation à ce que celle-ci signifie dans le monde réel.
Une théorie peut avoir plusieurs modèles.

\subsubsection{Seconde interprétation}

\begin{tabular}{lll}
$D_J$ : $\mathbb{N}$&&\\
$\text{val}_J(p)="p_J"$ &\hspace*{1cm} &$p_J : \mathbb{N}\rightarrow\mathbb{N} : d \rightarrow 2d$\\
$\text{val}_J(m)="m_J"$ &\hspace*{1cm} &$m_J : \mathbb{N}\rightarrow\mathbb{N} : d \rightarrow 3d$\\
$\text{val}_J(P)="P_J"$ &\hspace*{1cm} &$P_J(d_1,d_2)$ ssi $d_2=2d_1$ ou $d_2=3d_1$\\
$\text{val}_J(GP)="GP_J"$ &\hspace*{1cm} &$GP_J(d_1,d_2)$ ssi $d_2=4d_1$ ou $d_2=6d_1$\\
$\text{val}_J(GM)="GM_J"$ &\hspace*{1cm} &$GM_J(d_1,d_2)$ ssi $d_2=6d_1$ ou $d_2=9d_1$\\
\end{tabular}\\

Cette interprétation est également un modèle de \bsc{fam}.

\subsubsection{Exemple de preuve}

Voici une preuve dans la théorie \bsc{fam}.
Cette preuve est un exemple de l'approche syntaxique qui est mentionnée dans la section suivante.
La propriété qui est prouvée est vraie dans les deux interprétations.
On veut montrer  : 
$$\models_{\text{\bsc{fam}}} (\forall x)(\exists z) GM(x,z)$$

\begin{tabular}{lll}
1.&$(\forall x)(\forall y) \left(P(x,y)\Rightarrow GM(x, m(y)) \right)$&(Axiome de grand-mère)\\
2.&$(\forall x) \left(P(x,p(x))\Rightarrow GM(x, m(p(x))) \right)$&(Elimination de $\forall y$ et substitution $y/p(x)$)\\
3.&$(\forall x) P(x,p(x)) \Rightarrow (\forall x) GM(x,m(p(x)))$& (Distributivité $\forall/\Rightarrow$)\\
4.&$(\forall x) P(x,p(x))$ & (Axiome de père) \\
5.&$(\forall x) GM(x,m(p(x)))$&(Modus ponens)\\
6.&$(\forall x) (\exists y) GM(x,y)$&(Introduction de $\exists$)\\
\end{tabular}

Attention, cette preuve utilise le schéma de distributivité de $\forall$ sur $\Rightarrow$
dans la ligne (3).
Ce schéma dit que pour toutes formules $p$ et $q$
l'on peut remplacer $(\forall x) (p \Rightarrow q)$
par $(\forall x) p \Rightarrow (\forall x) q$.
Je vous laisse comme exercice de prouver la validité de ce schéma.
Indice: on peut utiliser l'approche sémantique, comme définie la section suivante.

\section{Propriétés des théories}

\begin{itemize}
\item Une formule fermée $p$ est \textit{valide} dans la théorie $Th$ si elle est vraie dans chaque modèle de $Th$. On écrit : 
$$\models_{Th} p$$
Soit l'ensemble des axiomes $Ax=\{Ax_1, \hdots, Ax_n\}$. On a bien que $\models_{Th} Ax_i$.\\
\item $q$ est une \textit{conséquence logique} de $p$ dans la théorie $Th$
si $q$ est vraie dans tous les modèles de $Th$ qui rendent $p$ vraie.
On écrit :  $$p \models_{Th} q$$
\item Une théorie est \textit{consistante} si elle a au moins un modèle ($>0$ modèles).
\item Une théorie est \textit{inconsistante} si elle n'a pas de modèle ($0$ modèles).\\
\end{itemize}

\subsection{Établir la validité}

Comment peut-on faire pour établir $\models_{Th} p$? Il y a deux approches différentes :
\begin{enumerate}
\item L'approche {\em sémantique} : on prend un modèle $I$ quelconque de $Th$ et on évalue $\textrm{VAL}_I (p)$
en utilisant le fait que $\text{VAL}_I (Ax_i)=T$.
\item L'approche {\em syntaxique} : on essaye de construire une preuve de $p$ à partir des axiomes, en appliquant les règles de $Th$.
Cette approche s'appelle aussi {\em théorie de la preuve}.\\
\end{enumerate}
En pratique, la deuxième approche est beaucoup plus souvent utilisée.\\

\subsection{Qualité d'une théorie}

Une théorie peut posséder certaines qualités qui la rend
meilleure que d'autres théories qui ne les possèdent pas.
Une ``bonne'' théorie est:
\begin {enumerate}
\item {\em consistante} : il est impossible de déduire \(p\) et \(\neg p\)  de la même théorie.
\item {\em minimale} : les axiomes sont indépendants: $\{Ax_1, \hdots, Ax_n\} \models Ax_k $\\
On peut vérifier la minimalité en construisant une interprétation $J$ telle que:\\
$\text{VAL}_J (Ax_k)= \mathrm{false}$\\
$\text{VAL}_J (Ax_i)= \mathrm{true}$ pour $i \neq k$
\item {\em complète} : les axiomes suffisent pour prouver les propriétés qui nous intéressent. Sinon il faut en ajouter.
\end {enumerate}

\subsubsection{Exemple dans l'astronomie}
\begin {enumerate}
\item {\textbf{Système de Copernic}} : Chaque axiome de chaque planète est indépendant, car elles tournent toutes autour du soleil.
\item{\textbf{Système de Ptolémée}} : Les axiomes de chaque planète dépendent de ceux de la Terre, car elle représente le centre de l'univers, mais elle est également une planète et dispose donc d'axiomes.
\end {enumerate}
